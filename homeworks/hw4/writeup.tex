\documentclass[12pt]{article}
\title{EE351K Homework 4}
\author{Hershal Bhave (hb6279)}
\date{Due September 25, 2014}

\usepackage{multicol}
\usepackage[in]{fullpage}
\usepackage{xcolor}
\usepackage{rotating}
\usepackage{mathtools}
\usepackage{amssymb}
\usepackage{cleveref}
\usepackage{graphics}
\usepackage{caption}
\usepackage{subcaption}
\usepackage[nosolutionfiles]{answers}
\usepackage[acronym]{glossaries}

\newenvironment{Ex}{\textbf{Problem}\vspace{.75em}\\}{}
\Newassociation{solution}{Soln}{Answers}
\pagebreak[3]
\newcommand{\Opentesthook}[2]{\Writetofile{#1}{\protect\section{#1: #2}}}
\renewcommand{\Solnlabel}[1]{\textbf{Solution}\quad}

\newcommand{\dd}[1]{\:\mathrm{d}{#1}}
\newcommand{\ddt}[1]{\frac{\dd{}}{\dd{#1}}}
\newcommand{\dddt}[1]{\frac{\dd{}^2}{\dd{#1}^2}}

\begin{document}
\maketitle
\begin{enumerate}
\item
  \begin{Ex}
    In a Galton Box, balls are released at the top of a triangle of
    pegs, say with $n$ layers. The pegs are aligned such that roughly
    with probability $p$ the ball goes to the right, otherwise it
    moves left, where typically $p = \frac{1}{2}$. The ball eventually
    lands in one of the bins at the bottom.
    \begin{enumerate}
    \item Under the above idealized model, what is the PMF for the
      random variable $X$ denoting the number of bin a typical ball
      would end up in?
    \item Watch a simulated demonstation of the physical system to
      see a Galton box where a bunch of balls are released. As the
      balls accumulate you start to see that the fraction in each bin,
      i.e., the empirical distribution associated with the experiment
      is quite close (but not always) to the true distribution
      computed earlier. How does the number of balls in the experiment
      affect the proximity of the empirical distribution to our model?
    \end{enumerate}
    \begin{solution} \hfill
      \begin{enumerate}
      \item The probability that the ball ends up in the $x$th bin
        from the left can be modeled by a binomial distribution:
        \begin{equation}
          \label{eq:1a-binomial-dist}
          P_X(x) = {n \choose x} p^k(1-p)^k
        \end{equation}
      \item The Law of Large Numbers dictates that as the sample size
        increases, the proximity of the measured system starts to
        approach that of the empirical system.
      \end{enumerate}
    \end{solution}
  \end{Ex}
\item
  \begin{Ex}
    The joint PMF of two random variables $X$ and $Y$ is given by
    \begin{equation*}
      \label{eq:2-question}
      P_{X,Y}(x,y) = \left\{
        \begin{aligned}
          & c \cdot (x^2 + y^2), && x=\{1,2,4\};\quad y=\{1,3\} \\
          & 0, && \text{otherwise}
        \end{aligned} \right
      \end{equation*}
      \begin{enumerate}
      \item Determine the constant $c$.
      \item What is $P(Y<X)$?
      \item What is $P(Y=X)$?
      \item Find the marginal PMFs of $X$ and $Y$. Are $X$ ,$Y$
        independent?
      \item Find the expectations $E[X]$, $E[Y]$, and $E[XY]$.
      \item Find the variances $\text{var}(X)$ and $\text{var}(X +Y)$.
      \item Let $A$ denote the event $X \ge Y$. Find $E[X|A]$ and
        $\text{var}(X|A)$.
      \end{enumerate}
      \begin{solution}
        {\huge TODO}
      \end{solution}
    \end{Ex}
  \item
    \begin{Ex}
      The joint PMF of two random variables $R$ and $S$ is given by
      \begin{equation*}
        \label{eq:3-question}
        P_{RS}(r,s) = \left\{
          \begin{aligned}
            \frac{4}{45} && r=1, s=1 \\
            \frac{6}{45} && r=1, s=2 \\
            \frac{6}{45} && r=1, s=3 \\
            \frac{6}{45} && r=2, s=1 \\
            \frac{9}{45} && r=2, s=2 \\
            \frac{9}{45} && r=2, s=3 \\
            \frac{2}{45} && r=3, s=1 \\
            \frac{3}{45} && r=3, s=2 \\
            0 && r=3, s=3 \\
          \end{aligned} \right
        Let A = \{S \not= 3\}, X = R+S, Y=R-S
      \end{equation*}
      \begin{enumerate}
      \item Find $p_S(s)$ and $p_{S|A}(s)$.
      \item Find $p_{R,Y}(r,y)$.
      \item Find $p_{X|A}(x)$.
      \end{enumerate}
      \begin{solution}
        {\huge TODO}
      \end{solution}
    \end{Ex}
  \item 
    \begin{Ex}
      The joint probability mass function of the random variables
      $X$, $Y$, $Z$ is
      \begin{equation*}
        \label{eq:4-question}
        \begin{aligned}
          p_{X,Y,Z}(0,0,0) = \frac{1}{9}, && p_{X,Y,Z}(0,0,1) =
          \frac{1}{9}, && p_{X,Y,Z}(0,1,0) = \frac{1}{18}, &&
          p_{X,Y,Z}(0,1,1) = \frac{2}{9} \\
          p_{X,Y,Z}(1,0,0) = \frac{1}{9}, && p_{X,Y,Z}(1,0,1) =
          \frac{1}{18}, && p_{X,Y,Z}(1,1,0) = \frac{1}{9}, &&
          p_{X,Y,Z}(1,1,1) = \frac{1}{9} \\
        \end{aligned}
      \end{equation*}
      \begin{enumerate}
      \item Find $E[XY\sqrt{Z}]$.
      \item Argue that $E[XY +XZ2 +YZ] = E[XY]+E[XZ2]+E[YZ]$ and then
        compute its value.
      \item Are $X$, $Y$ independent given $Z=1$? Compute $E[XY|Z = 1]$.
      \end{enumerate}
      \begin{solution}
        {\huge TODO}
      \end{solution}
    \end{Ex}
  \item
    \begin{Ex}
      Choose a number $X$ at random from the set of numbers $\{1, 2,
      3, 4\}$. Now choose a number at random from the subset no larger
      than $X$, that is, from $\{1,2,\ldots X\}$. Call this second
      number $Y$
      \begin{enumerate}
      \item Find the joint mass function of $X$ and $Y$.
      \item Find the conditional mass function of $X$ given that $Y =
        i$. Do it for $i = 1,2,\ldots ,4$. Also, find $E[Y]$.
      \item Are $X$ and $Y$ independent? Why?
      \end{enumerate}
      \begin{solution}
        {\huge TODO}
      \end{solution}
    \end{Ex}
  \item 
    \begin{Ex}
      The PMF for the result of any one roll of a three sided die with
      faces numbered $1$,$2$ and $3$ is
      \begin{equation*}
        p_X(x) = \left\{
          \begin{aligned}
            \frac{1}{2}, && x=1 \\
            \frac{1}{4}, && x=2 \\
            \frac{1}{4}, && x=3 \\
            0, && \text{otherwise} \\
          \end{aligned} \right
      \end{equation*}
      Consider a sequence of six independent rolls of this die, and
      let $X_i$ be the random variable corresponding to the $i$th
      roll.
      \begin{enumerate}
      \item What is the probability that exactly three of the rolls
        have result equal to 3?
      \item What is the probability that the first roll is 1, given
        that exactly two of the six rolls have result of 1?
      \item We are told that exactly three of the rolls resulted in 1
        and exactly three resulted in 2. Given this information, what
        is the probability that the sequence of rolls is 121212?
      \item Conditioned on the event that at least one roll resulted
        in 3, find the conditional PMF of the number of 3s.
      \end{enumerate}
    \end{Ex}
\end{enumerate}
\end{document}
