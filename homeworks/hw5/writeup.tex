\documentclass[12pt]{article}
\title{EE351K Homework 5}
\author{Hershal Bhave (hb6279)}
\date{Due October 9, 2014}

\usepackage{multicol}
\usepackage[in]{fullpage}
\usepackage{xcolor}
\usepackage{rotating}
\usepackage{mathtools}
\usepackage{amssymb}
\usepackage{cleveref}
\usepackage{graphics}
\usepackage{caption}
\usepackage{subcaption}
\usepackage[nosolutionfiles]{answers}
\usepackage[acronym]{glossaries}

\newenvironment{Ex}{\textbf{Problem}\vspace{.75em}\\}{}
\Newassociation{solution}{Soln}{Answers}
\pagebreak[3]
\newcommand{\Opentesthook}[2]{\Writetofile{#1}{\protect\section{#1: #2}}}
\renewcommand{\Solnlabel}[1]{\textbf{Solution}\quad}

\newcommand{\dd}[1]{\:\mathrm{d}{#1}}
\newcommand{\ddt}[1]{\frac{\dd{}}{\dd{#1}}}
\newcommand{\dddt}[1]{\frac{\dd{}^2}{\dd{#1}^2}}

\begin{document}
\maketitle
\begin{enumerate}
\item
  \begin{Ex}
    Let $Z$ be a continuous random variable with probability density
    function
    \begin{equation}
      \label{eq:1-question}
      f_Z(z) = \left\{
        \begin{aligned}
          & c(1+z^2), && \text{if }-2<z<1 \\
          & 0 && \text{otherwise}
        \end{aligned} \right.
    \end{equation}
    \begin{enumerate}
    \item For what value of c is this possible?
    \item Find the cumulative distribution function of Z.
    \end{enumerate}
    \begin{solution} \hfill
      \begin{enumerate}
      \item In this case, $c$ must be so that the area represented by
        the integral of \cref{eq:1-question} equals $1$.
        \begin{equation}
          \label{eq:1a-sol}
          \begin{aligned}
            1 &= \int_{-2}^{1} c(1+z) \dd{z} \\
            &= c\left[\frac{z^2}{2} + z\right]_{-2}^{1} \\
            &= c\left[\left(\frac{1}{2}+1\right) - \left(\frac{4}{2}+(-2)\right)\right] \\
            &= c\left(\frac{3}{2}\right) \\
            \implies c &= \frac{2}{3} \\
          \end{aligned}
        \end{equation}
      \item The CDF of $Z$ is is simply
        \begin{equation}
          \label{eq:1b-sol}
          \begin{aligned}
            \text{CDF}(Z) &= F_Z(z) \\
            F_Z(z) &= P(Z\le z) \quad \forall z \in \mathbb{R} \\
            &= \int_{-2}^{z} \left(\frac{2+2y}{3}\right) \dd{y} \\
          \end{aligned}
        \end{equation}
      \end{enumerate}
    \end{solution}
  \end{Ex}
\item
  \begin{Ex}
    Let X and Y be Gaussian random variables, with $X \sim N(0,1)$ and
    $Y \sim N(1,4)$.
    \begin{enumerate}
    \item Find $P(X \le 1.5)$ and $P(X \le -1)$.
    \item What is the distribution of $\frac{Y-1}{2}$?
    \item Find $P(-2 \le Y \le 1)$.
    \item Find $P(|Y|^2 < 1.5)$.
    \end{enumerate}
    \begin{solution} \hfill
      \begin{enumerate}
      \item A Gaussian (Normal) Random Variable is represented as
        \begin{equation}
          \label{eq:2-gaussian}
          f_X(x) = \frac{1}{\sqrt{2\pi\sigma^2}}e^{-\frac{(x-\mu)^2}{2\sigma^2}}
        \end{equation}
        The PDF of a Continuous Random Variable is simply the integral
        from $-\infty$.
        \begin{multicols}{2}
          \begin{equation}
            \label{eq:2ai-sol}
            \begin{aligned}
              P(X\le 1.5) &= \int_{-\infty}^{1.5} f_X(y) \dd{y} \\
              &= \int_{-\infty}^{1.5} \frac{1}{\sqrt{2\pi}}e^{-\frac{x^2}{2}}
              \dd{y} \\
              \implies P(X \le 1.5 )&= 0.9332.
            \end{aligned}
          \end{equation} &
          \begin{equation}
            \label{eq:2aii-sol}
            \begin{aligned}
              P(X\le -1) &= \int_{-\infty}^{-1} f_X(y) \dd{y} \\
              &= \int_{-\infty}^{-1} \frac{1}{\sqrt{2\pi}}e^{-\frac{x^2}{2}}
              \dd{y} \\
              \implies P(X \le -1 )&= 0.1587.
            \end{aligned}
          \end{equation}
        \end{multicols}
      \item Since $Y \sim N(1,4)$, we can declare that $\mu_Y = 1$ and
        $\sigma^2_Y = 4$. Assuming $Z = \frac{Y-1}{2}$
        \begin{equation}
          \label{eq:2b-z}
          \begin{aligned}
            Z &= \frac{Y-1}{2} \\
            \implies Z &= \left(\frac{1}{2}\right)Y + \left(-\frac{1}{2}\right) \\
          \end{aligned}
        \end{equation}
        From this we can define $a = \frac{1}{2}$ and $b = -\frac{1}{2}$.
        \begin{multicols}{2}
          \begin{equation}
            \label{eq:2b-e-sol}
            \begin{aligned}
              E[Z] &= a\mu + b \\
              &= \left(\frac{1}{2}\right)(1) + \left(-\frac{1}{2}\right) \\
              \implies E[Z] &= 0\\
            \end{aligned}
          \end{equation} &
          \begin{equation}
            \label{eq:2b-var-sol}
            \begin{aligned}
              Var(Z)&=a^2\sigma^2 \\
              &=\left(\frac{1}{2}\right)^2(4) \\
              \implies Var(Z)&=1 \\
            \end{aligned}
          \end{equation}
        \end{multicols}
        The distribution turns out to be standard normal.
      \item $P(2 \le Y \le 1)$ can be obtained through normalization
        \begin{equation}
          \label{eq:2c-sol}
          \begin{aligned}
            P(-2 \le Y \le 1) &= P(Y \le 1) - P(Y \le -2) \\
            &= P\left(Z\le \frac{y_0-\mu_y}{\sigma_y}\right) -
               P\left(Z\le \frac{y_1-\mu_y}{\sigma_y}\right) \\
            &= P(Z\le \frac{1-1}{2}) - P\left(Z\le \frac{-2-1}{2}\right) \\
            &= P(Z\le 0) - P\left(Z\le -\frac{3}{2}\right) \\
            &= P(Z\le 0) - \left(1 - P\left(Z\le \frac{3}{2}\right)\right) \\
            &= 0.5 - (1 - 0.9332) \\
            \implies P(-2 \le Y \le 1) &= 0.4332 \\
          \end{aligned}
        \end{equation}
      \item $P(|Y|^2 < 1.5)$ is captured the same as the above,
        through normalization
        \begin{equation}
          \label{eq:2d-sol}
          \begin{aligned}
            P(|Y|^2 \le 1.5) &= P(\sqrt{|Y|^2} \le \sqrt{1.5}) \\
            &= P(Y \le 1.2247) \\
            &= P\left(Z\le \frac{y_0-\mu_y}{\sigma_y}\right) \\
            &= P\left(Z\le \frac{1.2247-1}{2}\right) \\
            &= P\left(Z\le \frac{1.2247-1}{2}\right) \\
            &= P(Z\le 0.1123) \\
            \implies P(|Y|^2 \le 1.5) &= 0.5447 \\
          \end{aligned}
        \end{equation}
      \end{enumerate}
    \end{solution}
  \end{Ex}
\pagebreak[4]
\item
  \begin{Ex}
    Suppose that the cumulative distribution function of $X$ is given
    by
    \begin{equation}
      \label{eq:3-question}
      F_X(b) = \left\{
        \begin{aligned}
          & 0 && b<0 \\
          & \frac{b}{4} && 0 \le b < 1 \\
          & \frac{1}{2} + \frac{b-1}{4}  && 1 \le b < 2 \\
          & \frac{11}{12} && 2 \le b < 3 \\
          & 1 && 2 \le b \\
        \end{aligned} \right.
    \end{equation}
    \begin{enumerate}
    \item Find $P(X=i)$, for $i=1,2,3$.
    \item Find $P\left(\frac{1}{2} < X < \frac{2}{3}\right)$
    \end{enumerate}
    \begin{solution} \hfill
      \begin{enumerate}
      \item CDFs do not have values at any particular point; instead
        they are calculated as areas from $-\infty$ or a given
        value. Thus, the value at any specific $X=i$ is infinitesimal.
      \item Since we already have the CDF, we can simply use
        subtraction to find the the requested probability.
        \begin{equation}
          \label{eq:3b-sol}
          \begin{aligned}
            P\left(\frac{1}{2} < X < \frac{2}{3}\right) &=
            \frac{\frac{2}{3}}{4} - \frac{\frac{1}{2}}{4} \\
            \implies P\left(\frac{1}{2} < X < \frac{2}{3}\right) &=
            0.0417.
          \end{aligned}
        \end{equation}
      \end{enumerate}
    \end{solution}
  \end{Ex}
  \pagebreak[4]
\item
  \begin{Ex}
    A machine produces bolts the length of which obeys a normal
    distribution with mean 5 and standard deviation 0.2. A bolt is
    called defective if its length is not within a standard deviation
    of its mean.
    \begin{enumerate}
    \item What is the probability that a bolt produced by this machine
      is defective?
    \item What is the probability that among ten bolts none will be
      defective?
    \end{enumerate}
    \begin{solution} \hfill
      \begin{enumerate}
      \item We must find $P(X \ge 5.2) + P(X \le 4.8)$.
        \begin{equation}
          \label{eq:4a-sol}
          \begin{aligned}
            P(X \ge 5.2) + P(X \le 4.8) &= 1-(P(X \le 4.8) - P(X \ge
            5.2)) \\
            % &= 1-P(X \le 4.8) + P(X \ge 5.2) \\
            &= 1-P(X \le 4.8) + (1-P(X \le 4.8)) \\
            &= 2-2P(X \le 4.8) \\
            \implies P(X \ge 5.2) + P(X \le 4.8) &= 0.3714.
          \end{aligned}
        \end{equation}
      \item The probability that a bolt is not defective is the
        complement of our result in \cref{eq:4a-sol} ($0.6286$). The
        probability that all ten bolts will not be defective is
        \begin{equation}
          \label{eq:4b-sol}
          \begin{aligned}
            P(\text{none defective}) &= {10 \choose 10}
            (0.6826)^{10}(1-0.6826)^0 \\
            &= 0.6826^{10} \\
            \implies P(\text{none defective}) &= 0.0219 \\
          \end{aligned}
        \end{equation}
      \end{enumerate}
    \end{solution}
  \end{Ex}
\item
  \begin{Ex}
    A random variable $X$ is uniformly distributed between 0 and
    10. Find the probability that $X$ lies between a standard deviation
   $\sigma_X$ from its mean $\mu_X$.
    \begin{solution} \hfill \vspace{.75em}\\
      The standard deviation and mean of a Uniform distribution is as
      follows
      \begin{multicols}{2}
        \begin{equation}
          \label{eq:5-dev}
          \begin{aligned}
           \mu_X &= \frac{\theta_2 + \theta_1}{2} \\
           &= \frac{10}{2} \\
           \implies \mu_X &= 5 \\
          \end{aligned}
        \end{equation} &
        \begin{equation}
          \label{eq:5-mean}
          \begin{aligned}
           \sigma_X^2 &= \frac{(\theta_2 - \theta_1)^2}{12} \\
           &= \frac{100}{12} \\
           \implies \sigma_X^2 &= 8.3333 \\
           \implies \sigma_X &= 2.8868 \\
          \end{aligned}
        \end{equation}
      \end{multicols}
      The probability that $X$ lies in between the standard deviation
      from the mean is $0.5774$.
    \end{solution}
  \end{Ex}
\item
  \begin{Ex}
    Two random variables $X$ and $Y$ have the joint PDF given by
    \begin{equation}
      \label{eq:6-question}
      f_{XY}(x,y) = \left\{
        \begin{aligned}
          & ke^{-(2x+3y)}, && x \ge 0,\; y \ge 0 \\
          & 0, && \text{otherwise}
        \end{aligned} \right.
      \end{equation}
      \begin{enumerate}
      \item Find the value of the constant $k$ that makes $f_{XY}(x,
        y)$ a true joint PDF.
      \item Find the marginal PDFs of $X$ and $Y$.
      \item Find $P(X<Y<2)$
      \item Without actually performing the integration, obtain the
        integral that expresses the $P(X \le Y^2)$ (That is, just give
        the exact limits of the integration.)
      \item Find $E[X +Y]$.
      \end{enumerate}
      \begin{solution} \hfill
        \begin{enumerate}
        \item A true PDF, in general, has the property that the
          integral or sum over its entire support is $1$. We must
          check that propery here.
          \begin{equation}
            \label{eq:6a-sol}
            \begin{aligned}
              1 &= \int_0^\infty \int_0^\infty f_{X,Y}(x,y) \dd{y}
              \dd{x} \\
              &= k \int_0^\infty \int_0^\infty e^{-(2x+3y)} \dd{y}
              \dd{x} \\
              &= k \int_0^\infty
              \left[-\frac{1}{2}e^{-2x-3y}\right]_0^\infty
              \dd{x} \\
              &= k \int_0^\infty \frac{1}{2}e^{-3y} \dd{x} \\
              &= k \left[-\frac{1}{6}e^{-3y}\right]_0^\infty \dd{x} \\
              1 &= \frac{k}{6} \\
              \implies k &= 6.
            \end{aligned}
          \end{equation}
        \item The marginals are just the integrals of each individual
          random variable.
          \begin{multicols}{2}
            \begin{equation}
              \label{eq:5b-marginal-x}
              \begin{aligned}
                f_X(x) &= \int_0^\infty e^{-(2x+3y)} \dd{y} \\
                &= \left[-\frac{1}{3}e^{-(2x+3y)}\right]_0^\infty \\
                \implies f_X(x) &= \frac{1}{3}e^{-2x} \\
              \end{aligned}
            \end{equation} &
            \begin{equation}
              \label{eq:5b-marginal-y}
              \begin{aligned}
                f_Y(y) &= \int_0^\infty e^{-(2x+3y)} \dd{x} \\
                &= \left[-\frac{1}{2}e^{-(2x+3y)}\right]_0^\infty \\
                \implies f_Y(y) &= \frac{1}{2}e^{-3y} \\
              \end{aligned}
            \end{equation}
          \end{multicols}
        \item \hfill {\huge TODO}
        \item \hfill {\huge TODO}
        \end{enumerate}
    \end{solution}
  \end{Ex}
\item
  \begin{Ex}
    Show that for a positive continuous Random Variable $X$, $E[X] =
    \int P(X > x) \dd{x}$
    \begin{solution} \hfill
      {\huge TODO}
    \end{solution}
  \end{Ex}
\end{enumerate}
\end{document}
