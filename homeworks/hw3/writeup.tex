\documentclass[12pt]{article}
\title{EE351K Homework 3}
\author{Hershal Bhave (hb6279)}
\date{Due September 18, 2014}

\usepackage{multicol}
\usepackage[in]{fullpage}
\usepackage{xcolor}
\usepackage{rotating}
\usepackage{mathtools}
\usepackage{amssymb}
\usepackage{cleveref}
\usepackage{graphics}
\usepackage{caption}
\usepackage{subcaption}
\usepackage[nosolutionfiles]{answers}
\usepackage[acronym]{glossaries}

\newenvironment{Ex}{\textbf{Problem}\vspace{.75em}\\}{}
\Newassociation{solution}{Soln}{Answers}
\pagebreak[3]
\newcommand{\Opentesthook}[2]{\Writetofile{#1}{\protect\section{#1: #2}}}
\renewcommand{\Solnlabel}[1]{\textbf{Solution}\quad}

\newcommand{\dd}[1]{\:\mathrm{d}{#1}}
\newcommand{\ddt}[1]{\frac{\dd{}}{\dd{#1}}}
\newcommand{\dddt}[1]{\frac{\dd{}^2}{\dd{#1}^2}}

\begin{document}
\maketitle
\begin{enumerate}
\item
  \begin{Ex}
    The discrete random variable K has the following PMF:
    \begin{equation*}
      p_K(k) = \left\{
        \begin{aligned}
          b, && k=0 \\
          2b, && k=1 \\
          3b, && k=2 \\
          0, && \text{otherwise} \\
        \end{aligned}
      \end{equation*}
      \begin{enumerate}
      \item Find $b$.
      \item Determine the values of $P(K<2) \text{, } P(K \le 2),
        \text{ and } P(0<K<2)$
      \item Compute $E[K]$ and $\text{Var}(K)$.
      \item Let $Y = \frac{1}{1+K}$, find the PMF of $Y$.
      \item Compute $E[Y]$ and $E[Y^2]$.
      \end{enumerate}
      \begin{solution} \hfill
        \begin{enumerate}
        \item
          \begin{equation}
            \label{eq:1a-sol}
            \begin{aligned}
              1 &= b + 2b + 3b \\
              &= 6b \\
              b &= \frac{1}{6} \\
            \end{aligned}
          \end{equation}
        \item
          \begin{equation}
            \label{eq:1b-k-lt-2-sol}
            \begin{aligned}
              P(K<2) &= 2b + b \\
              &= 0.5 \\
              P(K\le2) &= b + 2b + 3b \\
              &= 1.0 \\
              P(0<K<2) &= 2b \\
              &= \frac{1}{3} \\
            \end{aligned}
          \end{equation}
        \item
          \begin{equation}
            \label{eq:1c-expected-sol}
            \begin{aligned}
              E[K] &= \sum_k k P_K(k) \\
              &= (0)(b) + (1)(2b) + 2(3b) \\
              &= 0 + \frac{1}{3} + 1 \\
              &= \frac{4}{3} \\
            \end{aligned}
          \end{equation}
          \begin{equation}
            \label{eq:1c-variance-sol}
            \begin{aligned}
              \text{var}(K) &= E[K^2] - (E[K])^2 \\
              &= \sum_K(k-E[K])^2p_K(k) \\
              &= \left(0 - \frac{4}{3}\right)^2(b) +
              \left(1 - \frac{4}{3}\right)^2(2b) +
              \left(2 - \frac{4}{3}\right)^2(3b) \\
              &= \left(-\frac{4}{3}\right)^2(b) +
              \left(-\frac{1}{3}\right)^2(2b) +
              \left(\frac{2}{3}\right)^2(3b) \\
              &= \frac{5}{9}
            \end{aligned}
          \end{equation}
        \item Let $g(K) = \frac{1}{1+K}$. Then we have $Y=g(K)$. The
          total encoding space of $y$ is $y = \{\frac{1}{3},
          \frac{1}{2}, 1\}$. We can acquire each value for $P_Y(y)$ by
          mapping the possible $y$ values back onto $P_Y(y)$:
          \begin{equation}
            \label{eq:1d-pmf}
            \begin{aligned}
              p_Y\left(\frac{1}{3}\right) &= \sum_{\{k | g(k) =
                \frac{1}{3}\}}p_K(k) && \implies k = 2 \\
              &= p_K(2) \\
              &= \frac{1}{2} \\
              p_Y\left(\frac{1}{2}\right) &= \sum_{\{k | g(k) =
                \frac{1}{2}\}}p_K(k) && \implies k = 1 \\
              &= p_K(1) \\
              &= \frac{1}{3} \\
              p_Y(1) &= \sum_{\{k | g(k) = 1\}}p_K(k) && \implies k = 0 \\
              &= p_K(0) \\
              &= \frac{1}{6} \\
            \end{aligned}
          \end{equation}
          Writing the PMF of $Y$ in a more easily organizable form:
          \begin{equation}
            \label{eq:1d-sol}
            p_Y(y) = \left\{
              \begin{aligned}
                \frac{1}{6}, && y = 1 \\
                \frac{1}{3}, && y = \frac{1}{2} \\
                \frac{1}{2}, && y = \frac{1}{3} \\
                0 && \text{otherwise} \\
              \end{aligned} \right
          \end{equation}
        \item
          \begin{equation}
            \label{eq:1e-e-y}
            \begin{aligned}
              E[Y] &= \sum_y y \cdot p_Y(y) \\
              &= \left(1\right)\left(\frac{1}{6}\right) +
              \left(\frac{1}{2}\right)\left(\frac{1}{3}\right) +
              \left(\frac{1}{3}\right)\left(\frac{1}{2}\right) \\
              &= \frac{1}{2} \\
            \end{aligned}
          \end{equation}
        \end{enumerate}
      \end{solution}
    \end{Ex}
  \item
    \begin{Ex}
      The number of $N$ calls arriving at a switchboard during a
      period of one hour has the PMF
      \begin{equation*}
        \begin{aligned}
          p_N(n) = \frac{10^ne^{-10}}{n!} \quad && n = 0, 1, \ldots
        \end{aligned}
      \end{equation*}
      \begin{enumerate}
      \item What is the probability that at least two calls arrive
        within one hour?
      \item What is the probability that at most three calls arrive
        within one hour?
      \item What is the probability that the number of calls that
        arrive within one hour is greater than three but less than or
        equal to six?
      \end{enumerate}
      \begin{solution} \hfill
        \begin{enumerate}
        \item The probability that at least two calls arrive within
          one hour can be found by the following
          \begin{equation}
            \label{eq:2a-desc}
            P(N \ge 2) = \sum_{n \ge 2}\frac{10^ne^{-10}}{n!}
          \end{equation}
          By property of PMFs, the probabilies will sum to $1$, so we
          can find $P(N \ge 2)$ by finding $1 - (P(0) + P(1))$.
          \begin{equation}
            \label{eq:2a-sol}
            \begin{aligned}
              P(N \ge 2) &= 1 - (P(0) + P(1)) \\
              &= 1 - \frac{10^0e^{-10}}{0!} - \frac{10^1e^{-10}}{1!} \\
              &= 1 - e^{-10} - 10e^{-10} \\
              &= 0.9995006008 \\
            \end{aligned}
          \end{equation}
        \item The probability that at most three calls arrive within
          one hour can be found by the following
          \begin{equation}
            \label{eq:1b-sol}
            \begin{aligned}
              P(N \le 3) &= P(0) + P(1) + P(2) + P(3) \\
              &= \frac{10^0e^{-10}}{0!} + \frac{10^1e^{-10}}{1!} +
              \frac{10^2e^{-10}}{2!} + \frac{10^3e^{-10}}{3!} \\
              &= e^{-10} + 10e^{-10} + \frac{10^2e^{-10}}{2} +
              \frac{10^3e^{-10}}{6} \\
              &= 0.01033605068 \\
            \end{aligned}
          \end{equation}
        \item The probability that the number of calls that
          arrive within one hour is greater than three but less than or
          equal to six can be found by the following
          \begin{equation}
            \label{eq:2c-sol}
            \begin{aligned}
              P(3 < N \le 6) &=  P(4) + P(5) + P(6) \\
              &= \frac{10^4e^{-10}}{4!} + \frac{10^5e^{-10}}{5!} +
              \frac{10^6e^{-10}}{6!} \\
              &= 0.119805370206 \\
            \end{aligned}
          \end{equation}
        \end{enumerate}
      \end{solution}
    \end{Ex}
  \item
    \begin{Ex}
      A man claims to have extrasensory perception. As a test, a fair
      coin is flipped 10 times, and the man is asked to predict the
      outcome in advance. He gets 7 out of the 10 correct. What is the
      probability that he would have done at least this well if he had
      no ESP?
      \begin{solution} \hfill
      \item  The question is simply asking ``What is the probability of
        getting 7 out of 10 coin guesses correct''. For this we will
        generate a Binomial Random Variable corresponding to the
        situation where the man guesses at least 7 out of 10 tosses
        correctly, $X$ $\sim$ Binomial(10, 0.5):
        \begin{equation}
          \label{eq:3-sol}
          \begin{aligned}
            P(X\ge7) &= p_X(7) + p_X(8) + p_X(9) + p_X(10) \\
            &= {10 \choose 7}(0.5)^{10} + {10 \choose 8}(0.5)^{10}
            + {10 \choose 9}(0.5)^{10} + (0.5)^{10} \\
            &= 0.171875000000 \\
          \end{aligned}
        \end{equation}
      \end{solution}
    \end{Ex}
  \item
    \begin{Ex}
      You toss independently a fair coin and you count the number of
      tosses until the first tail appears. If this number is $n$, you
      receive $2^n$ dollars. What is the expected amount that you will
      receive? How much would you be willing to pay to play this game?
      \pagebreak[2]
      \begin{solution} \hfill
      \item Let $g(X)$ be the amount of money received after $x$ coin
        tosses:
        \begin{equation}
          \label{eq:4-money}
          g(X) = 2^x
        \end{equation}
        The amount of tosses until the first tail appears can be
        modeled by a Geometric Random Variable, $X$ $\sim$
        Geometric($0.5$), such that its PMF can be modeled as
        \begin{equation}
          \label{eq:4-geometric}
          \begin{aligned}
            p_X(k) &= (0.5)^{k-1}(1-0.5) \\
            &= (0.5)^{k} \\
            &= 2^{-k} \\
          \end{aligned}
        \end{equation}
        The expected amount that you will receive can be modeled by
        the Expected Value Rule for Functions of Random Variables
        \begin{equation}
          \label{eq:4-expected-value}
          \begin{aligned}
            E[g(X)] &= \sum_x g(x)p_X(x) \\
            &= \sum_x 2^{-k} \cdot 2^k \\
          \end{aligned}
        \end{equation}
        Which turns out that
        \begin{equation}
          \label{eq:4-sol}
          E[g(X)] &= \sum 1 \\
          &= \infty \\
        \end{equation}
        The expected amount you would receive for guessing the correct
        coin toss number in infinite! Given unlimited resources, any
        amount of payment for this game would suffice.
      \end{solution}
    \end{Ex}
  \item
    \begin{Ex}
      An Internet service provider has $m$ modems with which it serves
      $n$ customers. Customers are not always active, so $m$ need not
      be greater than or equal to $n$. In fact, customers are only
      active with probability $p = 0.05$, and their activity is
      mutually independent. Let $X$ be a random variable denoting the
      total load on the system at a typical time, i.e., the total
      number of active customers in the system. Let $q$ denote the
      probability that the system is able to meet the current load,
      i.e., $q$ is a measure of the quality of service. If it is high
      users are more likely to be blocked. If it is low most users are
      likely to be able to access a modem. In this problem you will
      consider three types of problems associated with this types of
      scenario:
      \begin{enumerate}
      \item Suppose the Internet service provider has $n = 10$
        customers and $m = 4$ modems. what is the quality of service
        $q$ it can deliver to its customers?
      \item Suppose the Internet service provider has $n = 20$
        customers, how many modems $m$ does it need to ensure that the
        quality of service is at least $q = 0.9$?
      \item Suppose the Internet service provider has $m = 2$ modems,
        what is the maximum number of customers $n$ that it can support
        while ensuring a quality of service of at least $q = 0.2$?
      \end{enumerate}
      \begin{solution} \hfill
        \begin{enumerate}
        \item Assuming that the quality of service is a the
          probability of overload (not the probability of meeting the
          current load).

          In this case, the quality of service $q$ is the probability
          that the number of active customers $X$ will rise above
          $m=4$, i.e. $P(X>4)$, and signal an overload. Since each
          customer's activity is mutually independent, we can use a
          Binomial Random Variable, $X$ $\sim$ Binomial, to model the
          activity.
          \begin{equation}
            \label{eq:5-p-desc}
            P(X) = {10 \choose x} (0.05)^x(0.95)^{10-x}
          \end{equation}
          Thus, the probability that $X>4$ (overload condition) can be
          modeled by
          \begin{equation}
            \label{eq:5-p-sol}
            \begin{aligned}
              P(X>4) &= \sum_{i=5}^{10} {10 \choose i}
              (0.05)^i(0.95)^{10-i} \\
              &= 0.000063689831 \\
            \end{aligned}
          \end{equation}
        \item Solving for $m$
          \begin{equation}
            \label{eq:5-p-sol}
            \begin{aligned}
              P(X>m) &= \sum_{i=m+1}^{20} {20 \choose i}
              (0.05)^i(0.95)^{20-i} \\
              0.9 &\ge \sum_{i=m+1}^{20} {20 \choose i}
              (0.05)^i(0.95)^{20-i} \\
            \end{aligned}
          \end{equation}
        \item Solving for $n$
          \begin{equation}
            \label{eq:5-p-sol}
            \begin{aligned}
              P(X>2) &= \sum_{i=2+1}^{n} {n \choose i}
              (0.05)^i(0.95)^{n-i} \\
              0.2 &\ge \sum_{i=3}^{n} {n \choose i}
              (0.05)^i(0.95)^{n-i} \\
            \end{aligned}
          \end{equation}
        \end{enumerate}
      \end{solution}
    \end{Ex}
  \item
    \begin{Ex}
      Read the brief article on Zipf's Law which is posted in the
      course materials section in the Discrete RVs folder. Zipf's Law
      can be ``intuitively'' stated as follows: Suppose we have a
      countable collection of items and we label them $x = 1,2,
      \ldots$ based on their rank in terms of their popularity, i.e.,
      item $x$ is the $x$th most popular item. Under Zipf's Law with
      parameter $\alpha > 1$ the probability that the $i$th item is
      chosen/appears/is requested is inversely proportional to its
      rank to the power $\alpha$. In other words if $X$ denotes the
      rank of a random item following Zipf's law with parameter
      $\alpha$, then
      \begin{equation}
        \label{eq:zipfs-prob}
        p_X(x) \propto \frac{1}{x^\alpha} \text{ for } x = 1, 2, 3 \ldots
    \end{equation}
      This can be used to model the popularity of web sites, movies,
      songs on iTunes, uses of letters in the alphabet etc.
      \begin{enumerate}
      \item Find an expression in terms of $\alpha$ for the
        proportionality constant making $p_X$ a PMF when $\alpha =
        2$. Note: In general, one can show that for $\alpha \in [0,
        1]$ the proportionality constant does not exist and thus there
        is no PMF.
      \item What is the expected value of $X$ when $\alpha = 2$?
      \item Suppose there are a finite number of items, so $x =
        1,2,\ldots, n$ and now let $\alpha = 1$. Find an expression,
        in terms of $n$ for the proportionality constant for the PMF.
      \item How many times more likely is the most popular item versus
        the $x$th most popular item? For some large but finite $n$,
        compare this to the case where the popularity profile is given
        by a geometric distribution. Which distribution concentrates
        more probability on the higher ranked, i.e., lower $x$,
        values?
      \end{enumerate}
      \begin{solution} \hfill
        \begin{enumerate}
        \item Assuming that we are tasked to find a $k$ such that
          $p_X(x) = \frac{k}{x^\alpha}$, and satisfies
          $\sum_{x=1}^\infty p_X(x) = 1$ where $\alpha = 2$.
          This gives us
          \begin{equation}
            \label{eq:6a-reimann}
            \begin{aligned}
              1 &= \sum_{x=1}^\infty p_X(x) \\
              &= \sum_{x=1}^\infty \frac{k}{x^\alpha} \\
              &= k \sum_{x=1}^\infty x^{-\alpha}
            \end{aligned}
          \end{equation}
          The Reimann Zeta function is defined as
          \begin{equation}
            \label{eq:6a-reimann-def}
            \zeta(s) &= \sum_{n=1}^\infty n^{-s}
          \end{equation}
          which we will use to show that
          \begin{equation}
            \label{eq:6a-zeta}
            \begin{aligned}
              \zeta(2) &= \sum_{n=1}^\infty n^{-2} \\
              &= \frac{\pi^2}{6} \\
            \end{aligned}
          \end{equation}
          Continuing from \cref{eq:6a-reimann},
          \begin{equation}
            \label{eq:6a-sol}
            \begin{aligned}
              1 &= k \sum_{x=1}^\infty x^{-\alpha} \\
              &= k \cdot \frac{\pi^2}{6} \\
              \implies k &= \frac{6}{\pi^2} \\
            \end{aligned}
          \end{equation}
          The PMF does not exist for $\alpha \in [0,1]$ because the
          Reimann Zeta Function (reference \cref{eq:6a-reimann}) does
          not converge for values of $\alpha$ whose real component is
          less than or equal to $1$.
        \item The Expected Value of X is given by
          \begin{equation}
            \label{eq:6b-sol}
            \begin{aligned}
              E[X] &= \sum_x x \cdot p_X(x) \\
              &= \sum_{n=1}^\infty x \cdot x^{-2} \\
              &= \sum_{n=1}^\infty x^{-1} \\
              &= \infty
            \end{aligned}
          \end{equation}
        \item Assuming that there is a finite number of items, then we
          can ignore the fact that $\zeta(1)$ does not converge and
          continue from \cref{eq:6a-reimann}.
          \begin{equation}
            \label{eq:6c-sol}
            \begin{aligned}
              1 &= k \sum_{x=1}^n x^{-\alpha} \\
              \implies k &= \sum_{x=1}^n x \\
            \end{aligned}
          \end{equation}
        \item The most popular item is chosen with the probability given
          by $p_X(1) = \frac{k}{1^{\alpha}}$, and the $x$th most
          popular item's probability is given by
          $p_X(x) = \frac{k}{x^{\alpha}}$. We would like to find
          $\frac{p_X(1)}{p_X(x)}$, which is given by
          \begin{equation}
            \label{eq:6d-sol-reg}
            \begin{aligned}
              \frac{p_X(1)}{p_X(x)} &=
              \frac{\frac{k}{1^{\alpha}}}{\frac{k}{x^{\alpha}}} \\
              &= \frac{k}{1^{\alpha}} \frac{x^{\alpha}}{k} \\
              &= x^{\alpha}
            \end{aligned}
          \end{equation}
          If the profile was given by a geometric distribution, then
          the most popular item will be chosen with probability
          $p_X(1) = p$ and the $x$th most popular item will be chosen
          with probability $p_X(x) = (1-p)^{x-1}(p)$. We would like to
          find $\frac{p_X(1)}{p_X(x)}$, which is given by
          \begin{equation}
            \label{eq:6d-sol-geo}
            \begin{aligned}
              \frac{p_X(1)}{p_X(x)} &= \frac{p}{(1-p)^{x-1}(p)} \\
              &= \frac{1}{(1-p)^{x-1}} \\
              &= (1-p)^{1-x} \\
            \end{aligned}
          \end{equation}
        \end{enumerate}
      \end{solution}
    \end{Ex}
  \item
    \begin{Ex}
      I will show in class that, for a function $g : \mathbb{R}
      \rightarrow \mathbb{R}$, in general $E[g(X)] \not= g(E[X])$ for
      a random variable $X$. We did note that if $g$ was a linear
      function, however, this is indeed true. In this problem we
      consider this further in the case of convex functions. A
      function is convex if for all $x, y \in \mathbb{R}$ and for any
      $\lambda \in [0, 1]$ we have that
      \begin{equation}
        \label{eq:convex-prob}
        g(\lambda x + (1-\lambda)y) \le \lambda g(x) + (1-\lambda)g(y)
      \end{equation}
      If $-g$ is convex, then we say that $g$ is concave.
      \begin{enumerate}
      \item Use the above definition to draw a representative convex
        and concave function.
      \item Suppose $X$ is a discrete random variable that takes two
        values and $g$ is convex. Show that $E[g(X)] \ge g(E[X])$. It
        follows that if $g$ is concave then $E[g(X)] \le g(E[X])$.
      \item This result is true in general, i.e., for any kind of
        random variable. To convince yourself, can you extend this
        result to the case $X$ a discrete random variables that takes
        three values? Hint: you could perhaps condition...
      \end{enumerate}
      \begin{solution} \hfill
        \begin{enumerate}
        \item A representative convex function would be $g(x) = x^2$
          and a representative concave function would be $g(x) = -x^2$.
        \item For a convex $g(x)$,
          \begin{equation}
            \label{eq:7b-pre-sol}
            \begin{aligned}
              && E[g(X)] &&\ge&&& g(E[X]) \\
              \implies && \sum_x g(x) \cdot p_X(x) &&\ge&&&
              g\cdot\sum_x x \cdot p_X(x) \\
              \implies && g(x_1) \cdot p + g(x_2) \cdot (1-p) &&\ge&&&
              g(x_1 \cdot p + x_2 \cdot (1-p))
            \end{aligned}
          \end{equation}
          Which is essentially what was written in
          \cref{eq:convex-prob}. If $g(x)$ is concave, the $\ge$ is
          switched to a $\le$ in \cref{eq:7b-pre-sol}.
        \item
          \begin{equation}
            \label{eq:7c-pre-sol}
            \begin{aligned}
              && E[g(X)] &&\ge&&& g(E[X]) \\
              \implies && \sum_x g(x) \cdot p_X(x) &&\ge&&&
              g\cdot\sum_x x \cdot p_X(x) \\
              \implies && g(x_1) \cdot p_1 + g(x_2) \cdot p_2 + g(x_3)
              \cdot p_3 &&\ge&&& g(x_1 \cdot p_1 + x_2 \cdot p_2 + x_3
              \cdot p_3) \\
            \end{aligned}
          \end{equation}
          Which leads to
          \begin{equation}
            \label{eq:7c-sol}
            \begin{aligned}
              & g(p_1 + p_2)\cdot\left(\frac{p_1}{p_1 + p_2}x_1 +
                \frac{p_2}{p_1 + p_2}x_2 + p_3 \cdot x_3\right) \\
              \ge \quad & (p_1 + p_2)\cdot g\left(\frac{p_1}{p_1 +
                  p_2}x_1 + \frac{p_2}{p_1 + p_2}x_2\right) + p_3
              \cdot g(x_3) \\
            \end{aligned}
          \end{equation}
          And corresponds to \cref{eq:convex-prob}.
        \end{enumerate}
      \end{solution}
    \end{Ex}
  \item
    \begin{Ex}
      A utility function $u$ of some some quantity say $x$ is sometimes
      used to model how happy a person is with getting $x$. In the case
      where the allocation to is a random variable can interpret
      $E[u(X)]$ as the average utility the person will see for an
      allocation which is modeled by $X$. Given two random variables $X$
      and $Y$ if $E [u(X)] \ge E [u(Y)]$ we say that the person prefers
      the uncertainty associated with $X$ to $Y$ . If a persons utility
      function is concave we say the person is risk averse. Indeed
      recall that if $u$ is a concave function then
      $$ E[u(X)] \le u(E[X]) $$
      i.e., the person prefers to receive the average value vs the
      expected value of the utility. If it is convex then we call the
      person risk seeking, i.e.
      $$ E[u(X)] \ge u(E[X]) $$
      they prefer a scenario with uncertainty versus receiving the expected value.
      \begin{enumerate}
      \item Suppose you are given a choice between two `bets':
        \begin{itemize}
        \item You win \$100 with probability $0.9$ or \$0 with
          probability $0.1$.
        \item You win \$90 for sure.
        \end{itemize}
        Which one do you prefer? Are you risk
        seeking or risk averse?
      \item Suppose you are given a choice
        between two `bets':
        \begin{itemize}
        \item You lose \$100 with probability $0.9$ or \$0 with
          probability $0.1$.
        \item You lose \$90 for sure.
        \end{itemize}
        Now which one do you prefer? Are you risk seeking or risk averse?
      \item Most people will give different answers to Parts 1 and 2
        above. Which means that they ``view'' uncertainty in earnings
        differently than in losses. Think about some time in your life
        where you had to make such decisions, this could be associated
        with insurance, buying lottery tickets, etc..
      \end{enumerate}
      \begin{solution} \hfill
        \begin{enumerate}
        \item Win \$100 with probability $0.9$. This is risk-seeking.
        \item Lose \$100 with probability $0.9$. This is risk-seeking.
        \item Ok.
        \end{enumerate}
      \end{solution}
    \end{Ex}
  \end{enumerate}
\end{document}
