\documentclass[12pt]{article}
\title{EE351K Homework 2}
\author{Hershal Bhave (hb6279)}
\date{Due September 11, 2014}

\usepackage{multicol}
\usepackage[in]{fullpage}
\usepackage{xcolor}
\usepackage{rotating}
\usepackage{mathtools}
\usepackage{amssymb}
\usepackage{cleveref}
\usepackage{graphics}
\usepackage{caption}
\usepackage{subcaption}
\usepackage[nosolutionfiles]{answers}
\usepackage[acronym]{glossaries}

\newenvironment{Ex}{\textbf{Problem}\vspace{.75em}\\}{}
\Newassociation{solution}{Soln}{Answers}
\pagebreak[3]
\newcommand{\Opentesthook}[2]{\Writetofile{#1}{\protect\section{#1: #2}}}
\renewcommand{\Solnlabel}[1]{\textbf{Solution}\quad}

\newcommand{\dd}[1]{\:\mathrm{d}{#1}}
\newcommand{\ddt}[1]{\frac{\dd{}}{\dd{#1}}}
\newcommand{\dddt}[1]{\frac{\dd{}^2}{\dd{#1}^2}}

\begin{document}
\maketitle
\begin{enumerate}
\item
  \begin{Ex}
    What is the probability that at least one of a pair of fair six
    sided dice lands on 6, given that the sum of the dice is 9.
    \begin{solution}
      \hfill \vspace{0.75em} \\
      Let's declare the following:
      \begin{equation}
        \label{eq:1-var-declaration}
        \begin{aligned}
          A &= \text{ at least one die rolls lands on a }6 \\
          B &= \text{ the sum of the dice roll is }9
        \end{aligned}
      \end{equation}
      Meaning that
      \begin{equation}
        \label{eq:1-prob-declaration}
        P(A) = \frac{1}{6} \text{, } \quad
        P(B) = \frac{1}{9} \text{, } \quad
        P(A \cap B) = \frac{1}{18}
      \end{equation}
      Now obtaining $P(A | B)$ is trivial:
      \begin{equation}
        \label{eq:1-answer}
        \begin{aligned}
          P(A | B) &= \frac{P(A \cap B)}{P(B)} \\
          &= \frac{1}{2}
        \end{aligned}
      \end{equation}
    \end{solution}
  \end{Ex}
\item
  \begin{Ex}
    Alice and Bob play a game in which a hat is placed on their
    head. The hats can be one of two possible colors, say black or
    white at random. Each player can only see the color of the other
    player's hat, but not their own. The aim of the game is to guess
    the color of your own hat. To win the game they both have to guess
    their own hat color correctly.
    \begin{enumerate}
    \item Suppose both players guess the color of their hat
      independently. What is the probability that they win?
    \item Although the players cannot communicate during the game,
      they can agree on a strategy at the start of the game. Suppose
      the two players agree on guessing based on the assumption that
      they have the same color hat. What is the probability that they
      win?
    \end{enumerate}
    \begin{solution} \hfill
      \begin{enumerate}
      \item Let's declare the following:
        \begin{equation}
          \label{eq:2-var-declaration}
          \begin{aligned}
            A &= \text{ Alice guesses her color correctly} \\
            B &= \text{ Bob guesses his color correctly}
          \end{aligned}
        \end{equation}
        The probabilities then turn out to:
        \begin{equation}
          \label{eq:1-prob-declaration}
          P(A) = \frac{1}{2} \text{, } \quad
          P(B) = \frac{1}{2} \text{, } \quad
        \end{equation}
        The probability that they win turns out to be 
        \begin{equation}
          \label{eq:2a-answer}
          \begin{aligned}
            P(A \cap B) &= \frac{1}{2} \cdot \frac{1}{2} \\
            &= \frac{1}{4}
          \end{aligned}
        \end{equation}
      \item Assuming that Alice and Bob can see each others' hat color
        but cannot not communicate with each other. Alice and Bob
        cannot be correct if their hat colors differ in any way. Since
        they can see each others' hat colors, they can only be correct
        if both hats are the same color. Thus, the probability that
        they guess correctly is
        \begin{equation}
          \label{eq:2b-answer}
          P(A \cap B) = \frac{1}{2}
        \end{equation}
      \end{enumerate}
    \end{solution}
  \end{Ex}
\item
  \begin{Ex}
    We have two coins; one is fair and the second two headed. We pick
    one of the coins at random , we toss it twice and head shows both
    times. Find the probability that the coin picked is fair.
    \begin{solution} \hfill \\
      {\huge TODO}
    \end{solution}
  \end{Ex}
\item
  \begin{Ex}
    Suppose $\alpha$ percent of a certain population suffer from a
    serious disease. A person suspected of the disease is given two
    conditionally independent tests. Each test makes a correct
    diagnosis 90 percent of the time. i.e., if person has the disease
    the test is positive with probability $0.9$ and if the person does
    not have the disease the test results are negative with
    probability $0.9$.
    \begin{enumerate}
    \item For $\alpha = 0.1$, , find the probability that the person
      has the disease given that both tests are positive.
    \item Explain why the porbability is less even though the tests
      make a correct diagnosis 90 percent of the time.
    \item Suppose $\alpha = 10$, what do you think will happen to the
      probability that the person has the disease given that both
      tests are positive.
    \item Are the two tests independent? Explain.
    \end{enumerate}
    \begin{solution} \hfill \\
      {\huge TODO}
    \end{solution}
  \end{Ex}
\item
  \begin{Ex}
    At the end of each day Professor Plum puts her glasses in her
    drawer with probability $.90$, leaves them on the table with
    probability $.06$, leaves them in her briefcase with probability
    $0.03$, and she actually leaves them at the office with
    probability $0.01$. The next morning she has no recollection of
    where she left the glasses. She looks for them, but each time she
    looks in a place the glasses are actually located, she misses
    finding them with probability $0.1$, whether or not she already
    looked in the same place. (After all, she doesnt have her glasses
    on and she is in a hurry.)
    \begin{enumerate}
    \item Given that Professor Plum didnt find the glasses in her
      drawer after looking one time, what is the conditional
      probability the glasses are on the table?
    \item Given that she didnt find the glasses after looking for them
      in the drawer and on the table once each, what is the
      conditional probability they are in the briefcase?
    \item Given that she failed to find the glasses after looking in
      the drawer twice, on the table twice, and in the briefcase once,
      what is the conditional probability she left the glasses at the
      office?
    \end{enumerate}
    Note: Assume conditional independence of each search i.e., if $A$
    is the event that the glasses were not found after the first
    drawer search and B is the event that the glasses were not found
    after the second drawer search, then we have $P(A \cap B | C) =
    P(A|C)P(B|C)$ , if $P(C) > 0$.
    \begin{solution} \hfill \\
      {\huge TODO}
    \end{solution}
  \end{Ex}
  \item
    \begin{Ex}
      A committee consisting of three electrical engineers and three
      mechanical engi neers is to be formed at random from a group of
      seven electrical engineers and five mechanical engineers. Find
      the probability of the following events:
      \begin{enumerate}
      \item One particular electrical engineer must be on the committee.
      \item Two particular mechanical engineers cannot be together on
        the committee.
      \end{enumerate}
      \begin{solution} \hfill \\
        {\huge TODO}
      \end{solution}
    \end{Ex}
  \item
    \begin{Ex}
      This problem is aimed to help you refine your basic
      understanding of independence.
      \begin{enumerate}
      \item Suppose that an event $E$ is independent of itself. Show
        that either $P(E) = 0$ or $P(E) = 1$.
      \item Events $A$ and $B$ have probabilities $P(A)=0.3$ and
        $P(B)=0.4$. What is $P(A \cup B)$ if $A$ and $B$ are
        independent? What is $P(A \cup B)$ if $A$ and $B$ are
        mutually exclusive?
      \item Now suppose that $P(A) = 0.6$ and $P(B) = 0.8$. In this
        case, could the events $A$ and $B$ be independent? Could they
        be mutually exclusive?
      \item If $A$ and $B$ are independent, show that $A^c$ and $B$
        are also independent.
      \end{enumerate}
      \begin{solution} \hfill \\
        {\huge TODO}
      \end{solution}
    \end{Ex}
  \item
    \begin{Ex}
      Suppose Bob has $n$ keys, of which one will open his office
      door.
      \begin{enumerate}
      \item Suppose Bob tries the keys at random, discarding those
        that do not work, what is the probability that he will open
        the door on his $k\text{th}$ try?
      \item What is the probability that he will open the door if he
        does not discard previously tried keys?
      \end{enumerate}
      \begin{solution} \hfill \\
        {\huge TODO}
      \end{solution}
    \end{Ex}
  \item
    \begin{Ex}
      In the two networks shown in figure below, assume that the
      probability of each relay being closed is $p$ and that each
      relay is open or closed independently of any other relay. In
      each case find the probability that the current flows from $L$
      to $R$.

      (Hint : There are many ways of solving second network problem,
      one of them is to condition on Relay 3 and then use Total
      Probability Theorem.)
      \begin{solution} \hfill \\
        {\huge TODO}
      \end{solution}
    \end{Ex}
\end{enumerate}
\end{document}
