\documentclass[12pt]{article}
\title{EE360C Homework 1}
\author{Hershal Bhave (hb6279)}
\date{September 3, 2014}

\usepackage{multicol}
\usepackage[in]{fullpage}
\usepackage{xcolor}
\usepackage{rotating}
\usepackage{mathtools}
\usepackage{amssymb}
\usepackage{cleveref}
\usepackage[nosolutionfiles]{answers}
\usepackage[acronym]{glossaries}

\newenvironment{Ex}{\vspace{1.5em}\textbf{Problem}\\}{}
\Newassociation{solution}{Soln}{Answers}
\pagebreak[3]
\newcommand{\Opentesthook}[2]{\Writetofile{#1}{\protect\section{#1: #2}}}
\renewcommand{\Solnlabel}[1]{\textbf{Solution}\quad}

\newcommand{\dd}[1]{\:\mathrm{d}{#1}}
\newcommand{\ddt}[1]{\frac{\dd{}}{\dd{#1}}}
\newcommand{\dddt}[1]{\frac{\dd{}^2}{\dd{#1}^2}}

\begin{document}
\maketitle
\begin{enumerate}
\item
  \begin{Ex}
    Write down the sample space $\Omega$ for the following
    experiments\footnote{sample space: The set of all possible
      outcomes}:
    \begin{enumerate}
    \item A six sided die is rolled and the outcome is observed.
    \item The die is rolled sequentially until the first six appears.
    \end{enumerate}
    \begin{solution} \hfill
      \begin{enumerate}
      \item There are six potential outcomes, so the sample space
        $\Omega = \{1,2,3,4,5,6\}$.
      \item Assuming that the experiment is observing the outcome of
        the sequential die roll until a 6 is found, the experiment has
        only one potential outcome (6). Thus the sample space $\Omega
        = \{6\}$.
      \end{enumerate}
    \end{solution}
  \end{Ex}
\item
  \begin{Ex}
    A power cell consists of two subcells, each of which can provide
    voltage from 0 to 5 volts, regardless of what the other subcell
    provides. The power cell is functional if and only if the sum of
    the two voltages of the subcells is at least 6 volts. An
    experiment consists of measuring and recording the voltages of the
    two subcells.
    \begin{enumerate}
    \item Suggest a sample space $\Omega$ for the experiment.
    \item Using your proposed sample space, what is the event that the
      power cell is not functional but needs less than one additional
      volt to become functional.
    \end{enumerate}
    \begin{solution} \hfill
      \begin{enumerate}
      \item The sample space $\Omega = [0,5]$
      \item This assumes that the sum of both power subcells $a,b$ fall
        within the range $(5,6)$. Reference \cref{fig:power-fig} for a
        graphical representation. The event is simply the shaded
        region between the two curves, calculated to be $3.5/25$, or $0.14$.
        \begin{figure}[Hb]
          \centering
          \input{power-fig}
          \caption{A graphical representation of the event requirements}
          \label{fig:power-fig}
        \end{figure}
      \end{enumerate}
    \end{solution}
  \end{Ex}
\end{enumerate}

\end{document}