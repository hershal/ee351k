\documentclass[12pt]{article}
\title{EE360C Homework 1}
\author{Hershal Bhave (hb6279)}
\date{September 3, 2014}

\usepackage{multicol}
\usepackage[in]{fullpage}
\usepackage{xcolor}
\usepackage{rotating}
\usepackage{mathtools}
\usepackage{amssymb}
\usepackage{cleveref}
\usepackage[nosolutionfiles]{answers}
\usepackage[acronym]{glossaries}

\newenvironment{Ex}{\textbf{Problem}\vspace{.75em}\\}{}
\Newassociation{solution}{Soln}{Answers}
\pagebreak[3]
\newcommand{\Opentesthook}[2]{\Writetofile{#1}{\protect\section{#1: #2}}}
\renewcommand{\Solnlabel}[1]{\textbf{Solution}\quad}

\newcommand{\dd}[1]{\:\mathrm{d}{#1}}
\newcommand{\ddt}[1]{\frac{\dd{}}{\dd{#1}}}
\newcommand{\dddt}[1]{\frac{\dd{}^2}{\dd{#1}^2}}

\begin{document}
\maketitle
\begin{enumerate}
\item
  \begin{Ex}
    Write down the sample space $\Omega$ for the following
    experiments\footnote{sample space: The set of all possible
      outcomes}:
    \begin{enumerate}
    \item A six sided die is rolled and the outcome is observed.
    \item The die is rolled sequentially until the first six appears.
    \end{enumerate}
    \begin{solution} \hfill
      \begin{enumerate}
      \item There are six potential outcomes, so the sample space
        $\Omega = \{1,2,3,4,5,6\}$.
      \item Assuming that the experiment is observing the outcome of
        the sequential die roll until a 6 is found, the experiment has
        only one potential outcome (6). Thus the sample space $\Omega
        = \{6\}$.
      \end{enumerate}
    \end{solution}
  \end{Ex}
\item
  \begin{Ex}
    A power cell consists of two subcells, each of which can provide
    voltage from 0 to 5 volts, regardless of what the other subcell
    provides. The power cell is functional if and only if the sum of
    the two voltages of the subcells is at least 6 volts. An
    experiment consists of measuring and recording the voltages of the
    two subcells.
    \begin{enumerate}
    \item Suggest a sample space $\Omega$ for the experiment.
    \item Using your proposed sample space, what is the event that the
      power cell is not functional but needs less than one additional
      volt to become functional.
    \end{enumerate}
    \begin{solution} \hfill
      \begin{enumerate}
      \item The sample space $\Omega = [0,5]$
      \item This assumes that the sum of both power subcells $a,b$
        fall within the range $(5,6)$. Reference \cref{fig:power-fig}
        for a graphical representation. The event is the shaded region
        between the two curves, calculated to be $3.5/25$, or $0.14$.
        \begin{figure}[Hb]
          \centering
          \input{power-fig}
          \caption{A graphical representation of the event requirements}
          \label{fig:power-fig}
        \end{figure}
      \end{enumerate}
    \end{solution}
  \end{Ex}
\item 
  \begin{Ex}
    We will often use notation like $A:=\{(x,y) : 0 \le x, y\le 1\}$
    which means $A$ is a set of all pairs $(x,y)$ such that $0\le x\le
    1$ and $0 \le y \le 1$. Note that the values are assumed to be
    real valued unless otherwise specified. To develop a better
    understanding of the notations do sketch the following subsets of
    the x --- y plane.
    \begin{enumerate}
    \item $A_z := \{(x,y) : x^2+y^2 \le z^2\}$ for $z=5$.
    \item $B_z := \{(x,y) : (x,y) \in \mathbb{Z},\; x>0,\; y>0,\; x+y\ge z\}$ for $z=5$.
    \item $C := \{(x,y) : 0 \le y \le x \le 1\}$.
    \item $A_5 \cap B_3$.
    \item $D_z :=\{(x,y) : x \le z\}$ for $z=3$.
    \item $B_z \cup D_z$ for $z=3$.
    \end{enumerate}
    \begin{solution} \hfill
      {\huge TODO}
    \end{solution}
  \end{Ex}
\item 
  \begin{Ex}
    The following group of students are in a class: 24 male students
    of age over 18, 16 female students of age over 18, 12 male
    students of age under 18 and 8 female students of age under
    18. One student is choosen at random and the following events are
    defined A = Student is male, B = Student is female , C = Student is
    of age over 18 and D = Student is of age under 18. Evaluate the
    following:
    \begin{enumerate}
    \item $P(A \cup C)$.
    \item $P(B^c \cap D^c)$.
    \end{enumerate}
    \begin{solution} \hfill
      \begin{enumerate}
      \item $P(A \cup C) = \frac{52}{60} = 0.8667$
      \item $P(B^c \cap D^c) = \frac{24}{60} = 0.4$
      \end{enumerate}
    \end{solution}
  \end{Ex}
\item 
  \begin{Ex}
    Two events A and B have the following probabilities: $P(A) = 0.6$,
    $P(B) = 0.5$ and $P(A \cap B) = 0.3$. Calculate the following:
    \begin{enumerate}
    \item $P(A \cup B)$.
    \item $P(A^c \cup B^c)$.
    \end{enumerate}
    \begin{solution} \hfill
      \begin{enumerate}
      \item $P(A \cup B) = 0.8$
      \item $P(A^c \cup B^c) = 0.3$
      \end{enumerate}
    \end{solution}
  \end{Ex}
\item 
  \begin{Ex}
    Consider rolling a six-sided die. Let $A$ be the set of outcomes
    which are divisible by three. Let $B$ be the set of outcomes which
    are prime numbers.
    \begin{enumerate}
    \item Show that $(A^c \cup B^c)^c \cup (A^c \cup B)^c=A$ (Hint:
      Use DeMorgan's Law).
    \item Calculate and compare the sets on both sides of the previous
      equation.
    \end{enumerate}
    \begin{solution} \hfill
      \begin{enumerate}
      \item
        \begin{equation*}
          \begin{aligned}
             A &= (A^c \cup B^c)^c \cup (A^c \cup B)^c \\
             &= (A \cap B) \cup (A \cap B^c) \\
             A &= A
          \end{aligned}
        \end{equation*}
        $(A \cap B) \cup (A \cap B^c)$ is the same as $A$.
      \item $A = \{3,6\}$, $B = \{1,2,3,5\}$, and $\Omega = \{1,2,3,4,5,6\}$.

        From this information we can obtain $A^c = \{1,2,4,5\}$ and
        $B^c=\{4,6\}$.

        And now we can insert the sets into the equation:
        \begin{equation*}
          \begin{aligned}
            A &= (A^c \cup B^c)^c \cup (A^c \cup B)^c \\
            &= (\{1,2,4,5\} \cup \{4,6\})^c \cup (\{1,2,4,5\}
            \cup \{1,2,3,5\})^c \\
            &= (\{1,2,4,5,6\})^c \cup (\{1,2,3,4,5\})^c \\
            &= \{3\} \cup \{6\} \\
            \{3, 6\} &= \{3, 6\}
          \end{aligned}
        \end{equation*}
        The probabilities for both sides have been determined to be
        equivalent ($\frac{1}{3}$).
      \end{enumerate}
    \end{solution}
  \end{Ex}
\item
  \begin{Ex} For any three events A, B and C. Show that
    \begin{enumerate}
    \item $P(A \cap B) \ge P(A)+P(B) - 1$.
    \item $P(A \cup B \cup C) = P(A) + P(B) + P(C) - P(A \cap B) - P(A
      \cap C) - P(B \cap C) + P(A \cap B \cap C)$
    \end{enumerate}
    \begin{solution} \hfill
      \begin{enumerate}
      \item 
        \begin{equation*}
          \begin{aligned}
            P(A \cap B) &\ge P(A) + P(B) - 1 \\
            1 &\ge P(A) + P(B) - P(A \cap B) \\
            1 &\ge P(A) \cap P(B) \\
            1 &\ge P(A \cap B)
          \end{aligned}
        \end{equation*}
        This is true because the union of $A$ and $B$ can contain only
        contain a set less than or equal to the sample space, so the
        probability of $A$ or $B$ ocurring is less than or equal to
        the probability of hitting the entire sample space (1).
      \item
        \begin{equation*}
          \begin{aligned}
            P(A \cup B \cup C) &= P((A \cup B) \cup C) \\
            &= P((P(A) + P(B) - P(A \cap B)) \cup C) \\
            &= P(A) + P(C) - P(A \cap C) \\
            &\quad + P(B) + P(C) - P(B \cap C) \\
            &\quad - P(A \cap B) + P(C) - (P(A \cap B) \cap P(C)) \\
            &= P(A) + P(B) + P(C) - P(A \cap B) - P(A \cap C) - P(B
            \cap C) \\
            &\quad + P(A \cap B \cap C)
          \end{aligned}
        \end{equation*}

      \end{enumerate}
    \end{solution}
  \end{Ex}
\item 
  \begin{Ex}
    In class we informally defined a set as being countable if the
    elements of the set could be enumerated, say as
    $a_1,a_2,\ldots$. In other words, you can find an exhaustive way
    to list them. This problem is intended to help you refine your
    understanding of ``countable'' sets. You must provide clear
    succinct, preferably mathematical arguments.
    \begin{enumerate}
    \item Show that the positive rational numbers are countable – here
      we stick to positive rationals for simplicity. Recall that
      rational numbers are numbers that can be represented at the
      ratio of two integers, say $i/j$ where $i, j \in \{1, 2, 3,
      \ldots\}$. Hint: Find a way to visualize the rationals as a
      2-dimensional array, then think about how you can linearly
      enumerate the elements in that array.
    \item Suppose $A$ and $B$ are countable. Argue that the union $A
      \cup B$ is also countable.
    \item Consider the set $A$ containing all infinite sequences of 0s
      and 1s, e.g., $a = 10101010101\ldots$ is an element in
      $A$. Argue that $A$ is not countable. Hint: You can argue this
      by contradiction. Suppose you had an enumeration of all elements
      (sequences) in $A$, i.e., $a_1, a_2,\ldots$. Construct a
      sequence $a^∗$ which is not in the list. For example, $a^∗$ may
      differ from ai in the $i\text{th}$ digit for each $i = 1, 2,
      \ldots$. This is called a diagonal argument.
    \item Based on the claim made in the previous question, argue that
      the real numbers in $[0, 1]$ are uncountable.
    \end{enumerate}
    \begin{solution} \hfill
      {\huge TODO}
    \end{solution}
  \end{Ex}
\item 
  \begin{Ex}
    My goal in this class is to serve as your guide in learning the
    fundamental of probability and statistics and applications to
    engineering, but also to bring what you will learn what you will
    learn in my class into your daily life. To that end this last
    problem requires that you watch the following video
    http://www.gapminder.org/videos/the-joy-of-stats/ and answer the
    following questions.
    \begin{enumerate}
    \item According to the program what is the origin of the term
      ``statistics''?
    \item The program discusses correlation. Describe in your own
      words what they mean by correlation?
    \item The program explains a new paradigm for translation from one
      language to another. Explain in just a few sentences the
      previous versus the new approach.
    \item The program also explains a new paradigm that involves using
      simulation to generate (even more) data. Explain how this kind
      of simulated data can/is be used.
    \end{enumerate}
    \begin{solution} \hfill
      {\huge TODO}
    \end{solution}
  \end{Ex}
\end{enumerate}
\end{document}