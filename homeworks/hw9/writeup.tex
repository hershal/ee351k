\documentclass[12pt]{article}
\title{EE351K Homework 9}
\author{Hershal Bhave (hb6279)}
\date{Due November 13, 2014}

\usepackage{multicol}
\usepackage{float}
\usepackage[in]{fullpage}
\usepackage{xcolor}
\usepackage{mdframed}
\usepackage{tabularx}
\usepackage{rotating}
\usepackage{mathtools}
\usepackage{amssymb}
\usepackage{cleveref}
\usepackage{graphics}
\usepackage{caption}
\usepackage{wrapfig}
\usepackage{subcaption}
\usepackage[nosolutionfiles]{answers}
\usepackage[acronym]{glossaries}

\newenvironment{Ex}{\textbf{Problem}\vspace{.75em}\\}{}
\Newassociation{solution}{Soln}{Answers}
\pagebreak[3]
\newcommand{\Opentesthook}[2]{\Writetofile{#1}{\protect\section{#1: #2}}}
\renewcommand{\Solnlabel}[1]{\textbf{Solution}\quad}

\newcommand{\dd}[1]{\:\mathrm{d}{#1}}
\newcommand{\ddt}[1]{\frac{\dd{}}{\dd{#1}}}
\newcommand{\dddt}[1]{\frac{\dd{}^2}{\dd{#1}^2}}

\definecolor{silver}{rgb}{0.95,0.95,0.95}

\begin{document}
\maketitle
\begin{enumerate}
\item
  \begin{Ex}
    A modem transmits one million bits. Each bit is 0 or 1
    independently with equal probability. Estimate the probability
    that at least 502,000 ones have transmitted.
    \begin{solution} \hfill \vspace{.75em} \\
      We can use the Central Limit Theorem to calculate this
      probability.
      \begin{mdframed}[backgroundcolor=silver]
        \begin{description}
        \item[Central Limit Theorem] \hfill \vspace{.75em} \\
          $X_1, X_2,\ldots$ is a sequence of IID RVs with common $\mu$
          and $\sigma^2$. We define
          $$Z_n=\frac{\sum_{i=1}^n X_i -n\mu}{\sigma\sqrt{n}}$$
          So that
          $$\lim_{n\rightarrow\infty}\mathbf{P}(Z_n\le z) = \Phi(z)
          \quad\forall\; z$$
        \end{description}
      \end{mdframed}
      In our case, let $Y$ be the number of ones transmitted,
      i.e. $Y\sim\text{Binomial}(n,p)$, where $n=1000000$ and
      $p=0.5$. Then

      % Why put a table here and then a tabularx?
      % TODO: Investigate why this is
      \begin{table}[H]
        \centering
        \begin{tabularx}{\linewidth}{XX}
          \begin{equation}
            \label{eq:1-y-mean}
            \begin{aligned}
              E[Y] &= np \\
              \implies \mu_Y &= 500000 \\
            \end{aligned}
          \end{equation}
          &
          \begin{equation}
            \label{eq:1-y-var}
            \begin{aligned}
              \text{Var}(Y) &= np(1-p) \\
              \implies \sigma_Y^2 &= 250000 \\
            \end{aligned}
          \end{equation}
        \end{tabularx}
      \end{table}
      Using the Central Limit Theorem,
      \begin{equation}
        \label{eq:1-sol}
        \begin{aligned}
          P(\text{number of ones } \ge 502000)
          &= P\left(\sum X_i \ge 502000\right) \\
          &= P\left(\frac{\sum X_i - n\mu}{\sigma\sqrt{n}} \ge
            \frac{502000-n\mu}{\sigma\sqrt{n}}\right) \\
          &= P\left(Z_n\ge\frac{502000-n\mu}{\sigma\sqrt{n}}\right) \\
          &= 1 - P\left(Z_n\le \frac{502000-1000000 0.5}
            {\sqrt{0.25}\sqrt{1000000}}\right) \\
          \implies P(\text{number of ones } \ge 502000) &= 0.0000316712
        \end{aligned}
      \end{equation}
    \end{solution}
  \end{Ex}

\item
  \begin{Ex}
    On any given flight, an airline’s goal is to have the plane be as
    full as possible. For this reason, they sometimes choose to
    overbook.
    \begin{enumerate}
    \item If on average 10\% of customers cancel their tickets, and
      they do so independently of each other, what is the probability
      that a particular flight will be overbooked if the airline sells
      320 tickets for a plane that has a maximum capacity of 300
      people.
    \item What is the probability that a plane with maximum capacity
      of 150 people will be overbooked if the airline sells 160
      tickets.
    \item Given your answers on the previous questions, offer a
      comment on the relationship between the ability to overbook and
      the capacity of the airplane.
    \end{enumerate}
    \begin{solution} \hfill
      {\color{red} \huge TODO}
    \end{solution}
  \end{Ex}

\item
  \begin{Ex}
    A certain town has a Saturday night movie audience of 600 who must
    choose between two comparable movie theaters. Assume that the
    movie-going public is composed of 300 couples, and each couple
    independently flips a fair coin to decide which theatre to
    patronize.
    \begin{enumerate}
    \item Using the central limit theorem approximation, determine the
      minimum number of seats each theater must have so that the
      proability of exactly one theater running out of seats is less
      than 0.1.
    \item Repeat, assuming that each of the 600 customers makes an
      independent decision (instead of acting in pairs).
    \end{enumerate}
    \begin{solution} \hfill
      {\color{red} \huge TODO}
    \end{solution}
  \end{Ex}

\item
  \begin{Ex}
    To maintain a speed of v miles/hour in the presence of a headwind
    of speed $w$ mi/hr, a cyclist must generate a power output $y = 50
    + (v + w - 15)^3$ Watts. During each mile of road, the wind speed
    $W$ is a continuous uniform $(0,10)$ random variable that is
    independent of the wind speed in any other mile.
    \begin{enumerate}
    \item Lance the cyclist rides at constant velocity $v = 15$ mi/hr
      mile after mile. Let $Y$ denote Lance's power output over a
      randomly chosen mile. What is Lance's average power output
      $E[Y]$?
    \item What is the PDF $f_Y (y)$?
    \item Mile after mile, another cyclist Ashwin rides at a constant
      power $\hat{y}$ Watts in the presence of the same variable
      headwinds. Let $\hat{V}$ denote Ashwin's velocity over a
      randomly chosen mile. Ashwin chooses $\hat{y}$ so that
      $E[\hat{V}] = 16 \text{mi/hr}$. What is $\hat{y}$?
    \item Ashwin and Lance race across America (a 3000 mile race). Use
      the central limit theorem to estimate the probability $P[A]$
      that Ashwin wins.
    \end{enumerate}
    \begin{solution} \hfill
      {\color{red} \huge TODO}
    \end{solution}
  \end{Ex}

\item
  \begin{Ex}
    In a market survey, a company wishes to estimate the fraction of
    respondents who favor its product A. The poll is to be conducted
    with confidence level of 0.95, and the margin of error of 4
    percent. Assuming that surveying a respondent costs \$0.5, how
    large should the company’s budget be in order for the survey to
    meet the above criterion?

    Hint: Consider
    \begin{equation}
      \label{eq:5-hint}
      P(|M_n-a|<e)>c
    \end{equation}
    Where $e$ is the margin of error and $c$ is the confidence level.
    \begin{solution} \hfill
      {\color{red} \huge TODO}
    \end{solution}
  \end{Ex}

\item
  \begin{Ex}
    An airline burns $X$ gallons of jet fuel for each mile it travels. $X$
    is random, as it depends on atmospheric conditions, flight
    altitude, and wind in the upper atmosphere. We would like to
    define the efficiency of the plane in terms miles per gallon
    (MPG). Since the plane travels 1 mile using $X$ gallons of fuel, we
    can measures MPG as
    \begin{equation}
      \label{eq:6-mpg}
      \begin{aligned}
        \eta &= E\left[\frac{1}{X}\right] \text{ or } \\
        \eta^\prime &= \frac{1}{E[X]} \\
      \end{aligned}
    \end{equation}
    Since both $\eta$ and $\eta^\prime$ have the units of miles per gallon,
    which is the better measure of efficiency? Hint: Suppose
    $X_1,\ldots,X_m$ are IID where $X_i$ is the fuel consumed for mile
    $i$ of an $m$ mile trip. What is your trip MPG? What does the law
    of large numbers say?
    \begin{solution} \hfill
      {\color{red} \huge TODO}
    \end{solution}
  \end{Ex}

\item
  \begin{Ex}
    We have two boxes, each containing three balls: one black and two
    white in box 1; two black and one white in box 2. We choose one of
    the boxes at random, where the probability of choosing box 1 is
    equal to some given $p$, and then draw a ball.
    \begin{enumerate}
    \item Describe the MAP rule for deciding the identity of the box
      based on whether the drawn ball is black or white.
    \item Assuming that $p = 1$ , find the probability of an incorrect
      decision and compare it with the probability of error if no ball
      had 2 been drawn.
    \end{enumerate}
    \begin{solution} \hfill
      {\color{red} \huge TODO}
    \end{solution}
  \end{Ex}

\end{enumerate}
\end{document}
