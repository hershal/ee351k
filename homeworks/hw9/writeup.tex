\documentclass[12pt]{article}
\title{EE351K Homework 9}
\author{Hershal Bhave (hb6279)}
\date{Due November 13, 2014}

\usepackage{multicol}
\usepackage{float}
\usepackage[in]{fullpage}
\usepackage{xcolor}
\usepackage{mdframed}
\usepackage{tabularx}
\usepackage{rotating}
\usepackage{mathtools}
\usepackage{amssymb}
\usepackage{cleveref}
\usepackage{graphics}
\usepackage{caption}
\usepackage{wrapfig}
\usepackage{subcaption}
\usepackage[nosolutionfiles]{answers}
\usepackage[acronym]{glossaries}

\newenvironment{Ex}{\textbf{Problem}\vspace{.75em}\\}{}
\Newassociation{solution}{Soln}{Answers}
\pagebreak[3]
\newcommand{\Opentesthook}[2]{\Writetofile{#1}{\protect\section{#1: #2}}}
\renewcommand{\Solnlabel}[1]{\textbf{Solution}\quad}

\newcommand{\dd}[1]{\:\mathrm{d}{#1}}
\newcommand{\ddt}[1]{\frac{\dd{}}{\dd{#1}}}
\newcommand{\dddt}[1]{\frac{\dd{}^2}{\dd{#1}^2}}

\definecolor{silver}{rgb}{0.95,0.95,0.95}

\begin{document}
\maketitle
\begin{enumerate}
\item
  \begin{Ex}
    A modem transmits one million bits. Each bit is 0 or 1
    independently with equal probability. Estimate the probability
    that at least 502,000 ones have transmitted.
    \begin{solution} \hfill \vspace{.75em} \\
      We can use the Central Limit Theorem to calculate this
      probability.
      \begin{mdframed}[backgroundcolor=silver]
        \begin{description}
        \item[Central Limit Theorem] \hfill \vspace{.75em} \\
          $X_1, X_2,\ldots$ is a sequence of IID RVs with common $\mu$
          and $\sigma^2$. We define
          $$Z_n=\frac{\sum_{i=1}^n X_i -n\mu}{\sigma\sqrt{n}}$$
          So that
          $$\lim_{n\rightarrow\infty}\mathbf{P}(Z_n\le z) = \Phi(z)
          \quad\forall\; z$$
          The Central Limit Theorem states that the sum of a large
          number of IID RVs is approximately normal.
        \end{description}
      \end{mdframed}
      In our case, let $Y$ be the number of ones transmitted.
      % Why put a table here and then a tabularx?
      % TODO: Investigate why this is
      \begin{table}[H]
        \centering
        $Y\sim\text{Binomial}(n,p)$

        $n=1000000;\;p=0.5$
        \begin{tabularx}{\linewidth}{XX}
          \begin{equation}
            \label{eq:1-y-mean}
            \begin{aligned}
              E[Y] &= np \\
              \implies \mu_Y &= 500000 \\
            \end{aligned}
          \end{equation}
          &
          \begin{equation}
            \label{eq:1-y-var}
            \begin{aligned}
              \text{Var}(Y) &= np(1-p) \\
              \implies \sigma_Y^2 &= 250000 \\
            \end{aligned}
          \end{equation}
        \end{tabularx}
      \end{table}
      Using the Central Limit Theorem,
      \begin{equation}
        \label{eq:1-sol}
        \begin{aligned}
          P(\text{number of ones } \ge 502000)
          &= P\left(\sum X_i \ge 502000\right) \\
          &= P\left(\frac{\sum X_i - n\mu}{\sigma\sqrt{n}} \ge
            \frac{502000-n\mu}{\sigma\sqrt{n}}\right) \\
          &= P\left(Z_n\ge\frac{502000-n\mu}{\sigma\sqrt{n}}\right) \\
          &= 1 - P\left(Z_n\le \frac{502000-1000000 0.5}
            {\sqrt{0.25}\sqrt{1000000}}\right) \\
          \implies P(\text{number of ones } \ge 502000) &= 0.0000316712
        \end{aligned}
      \end{equation}
    \end{solution}
  \end{Ex}

\item
  \begin{Ex}
    On any given flight, an airline’s goal is to have the plane be as
    full as possible. For this reason, they sometimes choose to
    overbook.
    \begin{enumerate}
    \item If on average 10\% of customers cancel their tickets, and
      they do so independently of each other, what is the probability
      that a particular flight will be overbooked if the airline sells
      320 tickets for a plane that has a maximum capacity of 300
      people?
    \item What is the probability that a plane with maximum capacity
      of 150 people will be overbooked if the airline sells 160
      tickets?
    \item Given your answers on the previous questions, offer a
      comment on the relationship between the ability to overbook and
      the capacity of the airplane.
    \end{enumerate}
    \begin{solution} \hfill
      \begin{enumerate}
      \item
        We can model the probability that a single customer will show
        up to their booking by a Bernoulli RV.
        \begin{table}[H]
          \begin{equation}
            \label{eq:2a-rv}
            X_i\sim\text{Bernoulli}(0.9)
          \end{equation}
          \begin{tabularx}{\linewidth}{XX}
            \begin{equation}
              \label{eq:2a-mean}
              \begin{aligned}
                E[X_i] &= 0.9 \\
                \implies \mu_{X_i} &= 0.9 \\
              \end{aligned}
            \end{equation}
            &
            \begin{equation}
              \label{eq:2a-var}
              \begin{aligned}
                \text{Var}(X_i) &= (0.9)(1-0.9) \\
                \implies \sigma_{X_i}^2 &= 0.09 \\
              \end{aligned}
            \end{equation}
          \end{tabularx}
        \end{table}
        Using the Central Limit Theorem, we can find the probability
        that the flight will be overbooked, given the number of
        tickets sold ($n=320$) and the plane's capacity ($c=300$).
        \begin{equation}
          \label{eq:2a-sol}
          \begin{aligned}
            P(\text{passengers} > \text{capacity}) &=
            P\left(\sum_{i=1}^{n}X_i > c\right) \\
            &= P\left(\frac{\sum_{i=1}^{n}X_i - n\mu}{\sigma\sqrt{n}} >
              \frac{c-n\mu}{\sigma\sqrt{n}}\right) \\
            &= 1 - \Phi\left(\frac{c-n\mu}{\sigma\sqrt{n}}\right) \\
            &= 1 - \Phi\left(\frac{300-(320)(0.9)}
              {(\sqrt{0.09})(\sqrt{320})}\right) \\
            &= 1-\Phi(2.23607) \\
            \implies P(\text{passengers} > \text{capacity}) &= 0.0126737 \\
          \end{aligned}
        \end{equation}
      \item We can use \cref{eq:2a-sol} as inspiration, where $n=160$
        and $c=150$.
        \begin{equation}
          \label{eq:2b-sol}
          \begin{aligned}
            P(\text{passengers} > \text{capacity})
            &= 1 - \Phi\left(\frac{c-n\mu}{\sigma\sqrt{n}}\right) \\
            &= 1 - \Phi\left(\frac{150-(160)(0.9)}
              {(\sqrt{0.09})(\sqrt{160})}\right) \\
            &= 1 - \Phi(1.58114) \\
            \implies P(\text{passengers} > \text{capacity}) &=
            0.0569231 \\
          \end{aligned}
        \end{equation}
      \item It's harder to overbook a bigger plane.
      \end{enumerate}
    \end{solution}
  \end{Ex}

\item
  \begin{Ex}
    A certain town has a Saturday night movie audience of 600 who must
    choose between two comparable movie theaters. Assume that the
    movie-going public is composed of 300 couples, and each couple
    independently flips a fair coin to decide which theatre to
    patronize.
    \begin{enumerate}
    \item Using the central limit theorem approximation, determine the
      minimum number of seats each theater must have so that the
      proability of exactly one theater running out of seats is less
      than 0.1.
    \item Repeat, assuming that each of the 600 customers makes an
      independent decision (instead of acting in pairs).
    \end{enumerate}
    \begin{solution} \hfill
      \begin{enumerate}
      \item There are 600 people, but 300 couples are making
        decisions, where each couple can be modeled by a Bernoulli RV.
        \begin{table}[H]
          \begin{equation}
            \label{eq:3a-rv}
            X_i\sim\text{Bernoulli}(0.5)
          \end{equation}
          \begin{tabularx}{\linewidth}{XX}
            \begin{equation}
              \label{eq:3a-mean}
              \begin{aligned}
                E[X_i] &= 0.5 \\
                \implies \mu &= 0.5 \\
              \end{aligned}
            \end{equation}
            &
            \begin{equation}
              \label{eq:3a-var}
              \begin{aligned}
                \text{Var}(X_i) &= (0.5)(1-0.5) \\
                \implies \sigma^2 &= 0.25 \\
              \end{aligned}
            \end{equation}
          \end{tabularx}
        \end{table}
        We are tasked to find the probability that the deviation from
        the mean number of seats (i.e. probability that the mean
        amount of seats is not enough) is less than 0.1.
        \begin{equation}
          \label{eq:3a-prepresol}
          \begin{aligned}
            P(|M_n-\mu| > \delta) &< 0.1 \\
            P\left(\left|\frac{1}{n}\sum_{i=1}^nX_i-\mu\right|
              > \delta\right) &< 0.1 \\
            P\left(\left|\sum_{i=1}^nX_i-n\mu\right|>n\delta\right)&<0.1\\
            P\left(\left|\frac{\sum_{i=1}^nX_i-n\mu}{\sigma\sqrt{n}}\right|
              > \frac{n\delta}{\sigma\sqrt{n}}\right) &<0.1 \\
            P\left(Z_n>\frac{n\delta}{\sigma\sqrt{n}}\right)&<0.1\\
            1-P\left(Z_n\le\frac{n\delta}{\sigma\sqrt{n}}\right)&<0.1\\
            P\left(Z_n \le \frac{n\delta}{\sigma\sqrt{n}}\right) &< 0.9 \\
            \Phi\left(\frac{n\delta}{\sigma\sqrt{n}}\right) &< 0.9 \\
            \frac{n\delta}{\sigma\sqrt{n}} &= 1.28155 \\
            \delta &= \frac{1.28155\sigma}{\sqrt{n}} \\
            \implies\delta &= 0.0369952 \\
          \end{aligned}
        \end{equation}
        Where $\delta$ is the deviation from the mean. So the upper
        bound on the number of couples which can be seated with
        confidence 0.9 is
        \begin{equation}
          \label{eq:3a-presol}
          \implies (\mu+\delta)n = 161.099
        \end{equation}
        So the number of people that can be seated with confidence 0.9
        is 322.197.
      \item Taking inspiration from \cref{eq:3a-presol}
        \begin{equation}
          \label{eq:3b-presol}
          \begin{aligned}
            \delta &= \frac{1.28155\sigma}{\sqrt{n}} \\
            \implies\delta &= 0.0261596 \\
          \end{aligned}
        \end{equation}
        Where $\delta$ is the deviation from the mean. So the upper
        bound on the number of people which can be seated with
        confidence 0.9 is
        \begin{equation}
          \label{eq:3a-sol}
          \implies (\mu+\delta)n = 315.696
        \end{equation}
      \end{enumerate}
    \end{solution}
  \end{Ex}

\item
  \begin{Ex}
    To maintain a speed of v miles/hour in the presence of a headwind
    of speed $w$ mi/hr, a cyclist must generate a power output
    \begin{equation}
      \label{eq:4-power}
      y = 50 + (v + w - 15)^3 \text{ Watts}
    \end{equation}
    During each mile of road, the wind speed
    $W$ is a continuous uniform $(0,10)$ random variable that is
    independent of the wind speed in any other mile.
    \begin{enumerate}
    \item Lance the cyclist rides at constant velocity $v = 15$ mi/hr
      mile after mile. Let $Y$ denote Lance's power output over a
      randomly chosen mile. What is Lance's average power output
      $E[Y]$?
    \item What is the PDF $f_Y (y)$?
    \item Mile after mile, another cyclist Ashwin rides at a constant
      power $\hat{y}$ Watts in the presence of the same variable
      headwinds. Let $\hat{V}$ denote Ashwin's velocity over a
      randomly chosen mile. Ashwin chooses $\hat{y}$ so that
      $E[\hat{V}] = 16 \text{mi/hr}$. What is $\hat{y}$?
    \item Ashwin and Lance race across America (a 3000 mile race). Use
      the central limit theorem to estimate the probability $P(A)$
      that Ashwin wins.
    \end{enumerate}
    \begin{solution} \hfill
      \begin{enumerate}
      \item $v$ is fixed since its value is given.
        \begin{equation}
          \label{eq:4a-presol}
          \begin{aligned}
            E[Y] &= E[50 + (v+w-15)^3] \\
            &= 50 + E[W^3] \\
          \end{aligned}
        \end{equation}
        We must find $E[W^3]$, where $W\sim\text{Uniform}(0,10)$.
        \begin{equation}
          \label{eq:4a-w-cubed}
          \begin{aligned}
            E[g(W=w)] &= \int_{-\infty}^{\infty} (w^3)f_W(w) \dd{w} \\
            &= \int_{0}^{10}\frac{w^3}{10} \dd{w} \\
            &= \frac{1}{40} [w^4]_{0}^{10} \\
            &= \frac{10000}{40} \\
            \implies E[W^3] &= 250 \\
          \end{aligned}
        \end{equation}
        We can substitute this into \cref{eq:4a-presol} to obtain our
        solution:
        \begin{equation}
          \label{eq:4a-sol}
          \begin{aligned}
            E[Y] &= 50 + E[W^3] \\
            &= 50 + 250 \\
            \implies E[Y] &= 300 \\
          \end{aligned}
        \end{equation}
      \item For this problem we must use Change of Variables.
        \begin{mdframed}[backgroundcolor=silver]
          \begin{description}
          \item[Change of Variables] \hfill \vspace{.75em} \\
            If a certain dependent variable $y$ is a function of a
            single continuous and independent variable $x$, then:
            $$y = g(x)$$
            Assuming that $y=g(x)$ is a monotonically increasing
            function of $x$ with a unique inverse, then
            $$x=g^{-1}(y)$$
            The CDF of $Y$ can be obtained from the known CDF of $X$ as
            follows:
            \begin{equation*}
              \begin{aligned}
                F_Y(y) &= P(Y\le y) = P(X \le g^{-1}(y)) \\
                &= F_X(g^{-1}(y)) \\
                &= \int_{x|x\le g^{-1}(y)} f_X(x) \dd{x} \\
              \end{aligned}
            \end{equation*}
            If $x$ is monotonically decreasing then the CDF of $Y$ is
            $$F_Y(y) = 1 - F_X(g^{-1}(y))$$
          \end{description}
        \end{mdframed}
        In our case,
        \begin{equation}
          \label{eq:4b-setup}
          \begin{aligned}
            y &= g(w) = 50 + w^3  \quad\quad\text{and} \\
            w &= g^{-1}(y) = (y-50)^{1/3} \\
          \end{aligned}
        \end{equation}
        So now
        \begin{equation}
          \label{eq:4b-cdf}
          \begin{aligned}
            F_Y(y) &= \int_{w\le g^{-1}(y)} f_W(w) \dd{w} \\
            &= \int_{0}^{(y-50)^{1/3}} \frac{1}{10} \dd{w} \\
            \implies F_Y(y) &= \frac{(y-50)^{1/3}}{10} \\
          \end{aligned}
        \end{equation}
        Deriving the CDF will then give us the desired PDF.
        \begin{equation}
          \label{eq:4b-pdf}
          \begin{aligned}
            f_Y(y) &= \frac{d}{\dd{y}} F_Y(y) \\
            &= \frac{d}{\dd{y}} \frac{(y-50)^{1/3}}{10} \\
            \implies f_Y(y) &= \frac{1}{30 (y-50)^{2/3}}
          \end{aligned}
        \end{equation}
        We must now obtain the limits where this is valid. In our
        case,
        $$0\le w \le (y-50)^{1/3} \le 10$$
        We must find $y$ such that this inequality holds true.
        \begin{table}[H]
          \label{eq:4b-limits}
          \begin{tabularx}{\linewidth}{XX}
            \begin{equation}
              \begin{aligned}
                10 &\ge(y-50)^{1/3} \\
                1000 &\ge y-50 \\
                \implies y &\le 1050 \\
              \end{aligned}
            \end{equation}
            &
            \begin{equation}
              \begin{aligned}
                0 &\le (y-50)^{1/3} \\
                0 &\le y-50 \\
                \implies y &\ge 50 \\
              \end{aligned}
            \end{equation}
          \end{tabularx}
        \end{table}
        For precision,
        \begin{equation}
          \label{eq:4b-sol}
          \implies f_Y(y) = \left \{
            \begin{aligned}
              & \frac{1}{30 (y-50)^{2/3}} &&\quad 50\le y\le 1050 \\
              & 0 &&\quad\text{otherwise} \\
            \end{aligned} \right.
        \end{equation}
      \item It was given that $\hat{y}$ is constant and
        $E[\hat{V}]=16$. and we are tasked to find $\hat{y}$.
        \begin{equation}
          \label{eq:4c-sol}
          \begin{aligned}
            \hat{y} &= 50 + (\hat{v} + w - 15)^3 \\
            \hat{y} - 50 &= (\hat{v} + w - 15)^3 \\
            (\hat{y} - 50)^{1/3} &= \hat{v} + w - 15 \\
            &= E[\hat{V} + W - 15] \\
            &= E[\hat{V}] + E[W] - 15 \\
            &= 16+5-15 \\
            &= 6 \\
            \hat{y} - 50 &= 216 \\
            \implies \hat{y} &= 266 \\
          \end{aligned}
        \end{equation}
      \item
        Lance's velocity was previously given as 15 mi/hr. Ashwin's
        velocity, on the other hand, is variable with $E[\hat{v}] =
        16$. We know that Ashwin's power output is constant, however,
        which aids us in calculating the probability that Ashwin will
        win. The equation in \cref{eq:4-power} can be modified to
        compute Ashwin's velocity.
        \begin{equation}
          \label{eq:4d-ashwins-velocity}
          \begin{aligned}
            \hat{y} &= 50 + (\hat{v}+w-15)^3 \\
            \hat{y} - 50 &= (\hat{v}+w-15)^3 \\
            (\hat{y} - 50)^{1/3} &= \hat{v}+w-15 \\
            \hat{v} &= (\hat{y} - 50)^{1/3} + 15 - w \\
            &= (216)^{1/3} + 15 - w \\
            &= 6+15-w \\
            \implies \hat{v} &= 21-w \\
          \end{aligned}
        \end{equation}
        Since $W\sim\text{Uniform}(0,10)$, then that implies
        $\hat{V}\sim\text{Uniform}(11,21)$.

        In order to beat Lance, Ashwin must finish before
        $t=\frac{3000\text{mi}}{15\text{ mi/hr}}
        \implies t=200\text{ hrs}$.

        Ashwin's time to completion can be modeled by a Uniform RV.
        \begin{table}[H]
          \begin{equation}
            \label{eq:4d-time}
            T\sim\text{Uniform}\left(\frac{3000}{21},\frac{3000}{11}\right)
          \end{equation}
          \begin{tabularx}{\linewidth}{XX}
            \begin{equation}
              \label{eq:4d-mean}
              \begin{aligned}
                E[T] &= \frac{\frac{3000}{11} - \frac{3000}{21}}{2} \\
                \implies \mu_T &= 207.792 \\
              \end{aligned}
            \end{equation}
            &
            \begin{equation}
              \label{eq:4d-var}
              \begin{aligned}
                \text{Var}(T) &= \frac{(\frac{3000}{11} - \frac{3000}{21})^2}{12} \\
                \implies \sigma_T^2 &= 1405.520 \\
              \end{aligned}
            \end{equation}
          \end{tabularx}
        \end{table}
        Using the Central Limit Theorem, we can change the
        distribution of Ashwin's completion time to Normal instead of
        Uniform. We can use this new distribution to estimate the
        probability that Ashwin will beat Lance.
        \begin{equation}
          \label{eq:4d-sol}
          \begin{aligned}
            P(A) &= P(T\le 200) = P(T<200)\\
            &= P\left(\frac{T-\mu_T}{\sigma_T} \le
              \frac{200-\mu_T}{\sigma_T}\right) \\
            &= \Phi\left(\frac{-7.792}{37.4093}\right) \\
            &= 1 - \Phi(0.207846) \\
            \implies P(A) &= 0.417675 \\
          \end{aligned}
        \end{equation}
      \end{enumerate}
    \end{solution}
  \end{Ex}

\item
  \begin{Ex}
    In a market survey, a company wishes to estimate the fraction of
    respondents who favor its product A. The poll is to be conducted
    with confidence level of 0.95, and the margin of error of 4
    percent. Assuming that surveying a respondent costs \$0.5, how
    large should the company's budget be in order for the survey to
    meet the above criterion?

    Hint: Consider
    \begin{equation}
      \label{eq:5-hint}
      P(|M_n-a|<e)>c
    \end{equation}
    Where $e$ is the margin of error and $c$ is the confidence level.
    \begin{solution} \hfill
      First we must estimate the sample size. Modeling each $X_i$ as a
      Bernoulli distribution, we can assume that $\text{Var}(X_i) =
      \sigma^2 = 0.25$ as the highest possible value for the variance.
      \begin{equation}
        \label{eq:5-sol}
        \begin{aligned}
          &P(|M_n - \mu| < 0.04) > 0.95 \\
          &P\left(\left|\frac{\sum_{i=1}^n X_i - n\mu}{\sigma\sqrt{n}}\right| < \frac{(0.04)\sqrt{n}}{\sigma}\right) > 0.95 \\
          &\Phi\left(\frac{0.04\sqrt{n}}{\sigma}\right) > 0.95 \\
          &\frac{0.04\sqrt{n}}{\sigma} > \gamma \quad\text{Where } \gamma = 1.64485 \\
          &n = \left(\frac{\gamma\sigma}{0.04}\right)^2 \\
          \implies &n = 422.739 \\
        \end{aligned}
      \end{equation}
      Since it costs \$0.50 per person, we can expect it to cost
      around \$211.37.

      \subsubsection*{Alternate Solution}
      Using Chebyshev's formula, we can set the RHS
      equal to $\frac{\sigma_{M_n}^2}{d^2}$ where $d$ is the deviation from
      the mean (0.04).
      \begin{equation}
        \label{eq:5-alt-sol}
        \begin{aligned}
          \frac{\sigma_{M_n}^2}{d^2} &= \frac{\sigma^2}{4 n d^2} \\
          1-0.95 &= \frac{\sigma^2}{n d^2} \\
          n &= \frac{20\sigma^2}{d^2} \\
          \implies n &= 3125
        \end{aligned}
      \end{equation}
      So a conservative estimation of the cost using Chebyshev's rule
      \$1562.5.
    \end{solution}
  \end{Ex}

\pagebreak[4]
\item
  \begin{Ex}
    An airline burns $X$ gallons of jet fuel for each mile it travels. $X$
    is random, as it depends on atmospheric conditions, flight
    altitude, and wind in the upper atmosphere. We would like to
    define the efficiency of the plane in terms miles per gallon
    (MPG). Since the plane travels 1 mile using $X$ gallons of fuel, we
    can measures MPG as
    \begin{equation}
      \label{eq:6-mpg}
      \begin{aligned}
        \eta &= E\left[\frac{1}{X}\right] \text{ or } \\
        \eta^\prime &= \frac{1}{E[X]} \\
      \end{aligned}
    \end{equation}
    \begin{enumerate}
    \item Since both $\eta$ and $\eta^\prime$ have the units of miles
      per gallon, which is the better measure of efficiency?

      Hint: Suppose $X_1,\ldots,X_m$ are IID where $X_i$ is the fuel
      consumed for mile $i$ of an $m$ mile trip.
    \item What is your trip MPG?
    \item What does the law of large numbers say?
    \end{enumerate}
    \begin{solution} \hfill
      \begin{enumerate}
      \item $\eta$ is a better estimate because it preserves
        the original individual distributions instead of squashing the
        distributions into an average.
      \item The trip MPG is $\eta = E\left[\frac{1}{X}\right]$
      \item As $n\rightarrow\infty$, $\eta = \eta^\prime$
      \end{enumerate}
    \end{solution}
  \end{Ex}

\pagebreak[4]
\item
  \begin{Ex}
    We have two boxes, each containing three balls: one black and two
    white in box 1; two black and one white in box 2. We choose one of
    the boxes at random, where the probability of choosing box 1 is
    equal to some given $p$, and then draw a ball.
    \begin{enumerate}
    \item Describe the MAP rule for deciding the identity of the box
      based on whether the drawn ball is black or white.
    \item Assuming that $p = 1$, find the probability of an incorrect
      decision and compare it with the probability of error if no ball
      had been drawn.
    \end{enumerate}
    \begin{solution} \hfill
      \begin{enumerate}
      \item
        $\Theta = \text{ box number}(1,2)$; $X = \text{ ball color}$.
        We can describe the MAP rule as follows
        \begin{equation}
          \label{eq:7a}
          P_{\Theta|X}(\theta|x) = \frac{p_\Theta(\theta)P_{X|\Theta}(x|\theta)}{P_X(x)}
        \end{equation}
        or alternatively
        \begin{equation}
          \label{eq:7a-alt}
          P_{\Theta|X}(\theta|x) =
          \frac{p_\Theta(\theta)p_{X|\Theta}(x|\theta)}
          {\sum_{\theta^{\prime}}p_\Theta(\theta^\prime)
            P_{X|\Theta}(x|\theta^\prime)}
        \end{equation}
        Where
        \begin{table}[H]
          \centering
          \begin{tabularx}{\linewidth}{XX}
            \begin{equation}
              \label{eq:7a-val-1}
              p_{X|\Theta}(x|1) = \left\{
                \begin{aligned}
                  &1/3 &&\quad x=\text{black} \\
                  &2/3 &&\quad x=\text{white} \\
                \end{aligned}\right.
            \end{equation}
            &
            \begin{equation}
              \label{eq:7a-val-2}
              p_{X|\Theta}(x|1) = \left\{
                \begin{aligned}
                  &2/3 &&\quad x=\text{black} \\
                  &1/3 &&\quad x=\text{white} \\
                \end{aligned}\right.
            \end{equation}
          \end{tabularx}
        \end{table}
      \item The probability of an incorrect decision can be modeled by
        the equation
        \begin{equation}
          \label{eq:7b-map}
          p_{X|\Theta}(x|1)P_\Theta(1) > p_{X|\Theta}(x|2)p_\Theta(2)
        \end{equation}
        The probability of an incorrect decision if no ball had been
        drawn is exactly .5 since there are an equal number of balls
        in total.
      \end{enumerate}
    \end{solution}
  \end{Ex}

\end{enumerate}
\end{document}
