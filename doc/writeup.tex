\documentclass[12pt]{book}
\title{The EE351K Manual}
\author{Hershal Bhave (hb6279)}
\date{Updated \today}

\usepackage{multicol}
\usepackage[in]{fullpage}
\usepackage{xcolor}
\usepackage{rotating}
\usepackage{mathtools}
\usepackage{amssymb}
\usepackage{cleveref}
\usepackage[nosolutionfiles]{answers}
\usepackage[acronym]{glossaries}

\newcommand{\dd}[1]{\:\mathrm{d}{#1}}
\newcommand{\ddt}[1]{\frac{\dd{}}{\dd{#1}}}
\newcommand{\dddt}[1]{\frac{\dd{}^2}{\dd{#1}^2}}

\newenvironment{Ex}{\vspace{1.5em}\hspace{-1.5em}\textbf{Problem}\\}{}
\Newassociation{solution}{Soln}{Answers}
\pagebreak[3]
\newcommand{\Opentesthook}[2]{\Writetofile{#1}{\protect\section{#1: #2}}}
\renewcommand{\Solnlabel}[1]{\textbf{Solution}\quad}

\makeglossaries
\begin{document}
\maketitle
\tableofcontents
\chapter{Introduction}
\section{Interpretations}
There are two interpretations for probability:
\begin{description}
\item[Frequency of Occurance] the precent of successes in a moderaly
  large set of similar experiments.
\item[Subjective Belief] an experts' opinion
\end{description}

\section{Deterministic conclusions about uncertain phenomena}
The Law of Large Numbers dictates that the sample averages converge to
true averages as the sample size increases. The Central Limit Theorem
describes the distributino of $\sqrt{n}S_n$ approaches a bell-shaped
curve...

\section{Faulty Intuition}
\begin{Ex}
  Lattery that cnosists of three numbers chosen at random from 0 to
  50. Joe notices that the average of the winning numbers is 25, so he
  concludes that he should bet only numbers that have an average of 25.
\end{Ex}
This is faulty intuition because each number is equally as likely to
pick as the next.

\chapter{Review of Set Theory}
A set is a collection of objects, which are the elements of the set. A
set with no elements is referred to as the empty set ($\emptyset$).
\section{Describing Sets}
The elements can be listed as such
$$A = \{x_1, x_2, \ldots, x_n\}$$

We can also describe properties of sets as such
$$A = \{ x | x \text{ satisfies } p\}$$

Sets can also contain each other and describe equality
$$A = B \Rightarrow A \subset B \text{ and } B \subset A$$

Sets can be countable or uncontable, where we can count the elements
of the (possibly infinite) set (countable). Likely, relations
involving integers or real numbers are countable.

\section{Operations on Sets}
Union, Intersection, Compliment, Subtraction

\section{Algebra of Sets}
\subsection{Associativity}
$A \cup B \cup C$ is equivalent to $A \cup (B \cup C)$, which is
equivalent to $(A \cup B) \cup C$, which is equivalent to $(A \cup C)
\cup B$.
\subsection{Distributive}
$A \cup B \cup C$ is equivalent to $A \cup (B \cup C)$, which is
equivalent to $(A \cup B) \cup C$, which is equivalent to $(A \cup C)
\cup B$.

\printglossaries
\end{document}